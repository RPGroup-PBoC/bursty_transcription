% !TEX root = ./busty_transcription.tex
\begin{abstract}
The study of transcription remains one of the centerpieces of modern biology with implications in settings from development to metabolism to evolution to disease.   Precision measurements
using a host of different techniques including fluorescence and sequencing readouts have
raised the bar for what it means to quantitatively understand transcriptional
regulation.  The simplest example of a regulatory architecture is the repressor-operator
model originally posited by Jacob and Monod in the 1960s, in which a single repressor binds to a promoter site and inhibits transcription.  Our understanding
of this model architecture is sufficiently refined both experimentally
and theoretically that it has become possible to carefully discriminate between different
conceptual pictures of how simple repression works.  In this paper, we show how seven
distinct models of the simple-repression motif, based both on equilibrium and kinetic
thinking, can be used to derive the predicted level of mean expression and its variance.
These different models are invoked to confront a variety of different data on both
mean and variance in gene expression, illustrating the extent to which such
models can be distinguished.
\end{abstract}