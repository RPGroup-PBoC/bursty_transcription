% !TEX root = ./busty_transcription.tex
\section{Means}\label{section_02_means}

As noted in the previous section, there are two broad classes of models in play
for computing the input-output functions of regulatory architectures as shown in
Figure~\ref{fig1:means_cartoons}. In both classes of model, the promoter is
imagined to exist in a discrete set of states of occupancy, with each such state
of occupancy accorded its own rate of transcription -- including no 
transcription for many of these states. The models are probabilistic with each
state assigned some probability and the overall rate of transcription given by 
\begin{equation}
\mbox{average rate of transcription} = \sum_i r_i p_i,
\end{equation}
where $i$ labels the distinct states, $p_i$ is the probability of the
$i^{\text{th}}$ state, and $r_i$ is the rate of transcription of that state.
Ultimately, the different models differ along several key axes: what states to
consider and how to compute the probabilities of those states.

The first class of models that are the focus of the present section on
predicting the mean level of gene expression, sometimes known as thermodynamic
models, invoke the tools of equilibrium statistical mechanics to compute the
probabilities~\cite{Ackers1982, Shea1985, Buchler2003, Vilar2003a, Vilar2003b,
Bintu2005a, Bintu2005c, Gertz2009, Sherman2012, Saiz2013}. As seen in
Figure~\ref{fig1:means_cartoons}(B), even within the class of thermodynamic
models, we can make different commitments about the underlying microscopic
states of the promoter.  Indeed, the list of options considered here does not at
all exhaust the suite of different microscopic states we can assign to the
promoter.

The second class of models that allow us to access the mean gene expression use
chemical master equations to compute the probabilities of the different
microscopic states ~\cite{Ko1991, Peccoud1995, Record1996, Kepler2001,
Sanchez2008, Shahrezaei2008, Sanchez2011, Michel2010}. As seen in
Figure~\ref{fig1:means_cartoons}(C), we consider a host of different
nonequilibrium models, each of which will have its own result for both the mean
(this section) and noise (next section) in gene expression.

%The second class of models based on chemical master equations compute the time-evolution
%of the probability distribution for mRNA.

\subsection{Fold-changes are indistinguishable across models}
As a first stop on our search for the ``right'' model of simple repression, let
us consider what we can learn from theory and experimental measurements on the
average level of gene expression in a population of cells. One experimental
strategy that has been particularly useful (if incomplete since it misses out on
gene expression dynamics) is to measure the fold-change in mean expression. The
fold-change is defined as
\begin{equation}
\text{fold-change}
= \frac{\langle \text{gene expression with repressor present} \rangle}
        {\langle \text{gene expression with repressor absent} \rangle}
= \frac{\langle m (R > 0) \rangle}{\langle m (R = 0) \rangle}
\label{eq:fc_def}
\end{equation}
where angle brackets $\left\langle \cdot \right\rangle$ denote the average over
a population of cells and mean mRNA $\langle m\rangle$ is viewed as a function
of repressor copy number $R$. What this means is that the fold-change in gene
expression is a relative measurement of the effect of the transcriptional 
repressor ($R > 0$) on the gene expression level compared to an unregulated
promoter ($R = 0$). The second equality in~\ref{eq:fc_def} follows from assuming
that the translation efficiency, i.e., the number of proteins translated per
mRNA, is the same in both conditions. In other words, we assume that mean
protein level is proportional to mean mRNA level, and that the proportionality
constant is the same in both conditions and therefore cancels out in the ratio.
This is reasonable since the cells in the two conditions are identical except
for the presence of the transcription factor, and the model assumes that the
transcription factor has no direct effect on translation.

Fold-change has proven a very convenient observable in past
work~\cite{Garcia2011a, Razo-Mejia2018, Chure2019}. Part of its utility in
dissecting transcriptional regulation is its ratiometric nature, which removes
many secondary effects that are present when making an absolute gene expression
measurement. Also, by measuring otherwise identical cells with and without a
transcription factor present, any biological noise common to both conditions can
be made to cancel away. In addition, the presence of different operator sites,
i.e. binding sites for the repressor, is known to change the basal expression
level in the absence of repressor~\mmnote{look up cite}\mrm{does reg-seq mention
anything like this in their SI? I definitely saw that on my own experiments, but
didn't mention it in the paper}. Computing fold-changes neatly cancels this
effect as well.

\fig{fig1:means_cartoons} depicts a smorgasbord of mathematicized cartoons for
simple repression using both thermodynamic and kinetic models that have appeared
in previous literature. For each cartoon, we calculate the fold-change in mean
gene expression as predicted by that model, deferring some algebraic details to
Appendix~\ref{sec:non_bursty}. What we will find is that all cartoons collapse
to a single master curve, shown in \fig{fig1:means_cartoons}(D), which contains
just two parameters. We label the parameters $\Delta F_R$, an effective free
energy parametrizing the repressor-DNA interaction, and $\rho$, which subsumes
all details of transcription in the absence of repressors. We will offer some
intuition for why this master curve exists and discuss why at the level of the
mean expression, we are unable to discriminate ``right'' from ``wrong'' cartoons
given only measurements of fold-changes in expression.

\subsubsection{The Two-State Equilibrium Model}
In this simplest model, depicted as (1) in Figure~\ref{fig1:means_cartoons}(B),
the promoter is idealized as existing in one of two states, either repressor
bound or repressor unbound. The rate of transcription, and therefore the level
of gene expression, is assumed to be proportional to the fraction of time spent
in the repressor unbound state. Mathematically, this amounts to the statement
that the mean mRNA level is given by
\begin{equation}
\langle m \rangle = \frac{r}{\gamma}
        \left(1 + \frac{R}{N_{NS}} e^{-\beta\Delta\varepsilon_R}\right)^{-1},
\end{equation}
where $r$ is the transcription rate from the repressor unbound state, $\gamma$
is the mRNA degradation rate, $R$ is repressor copy number, $N_{NS}$ is the
number of nonspecific binding sites in the genome where repressors spend most
of their time when not bound to the operator, $\beta \equiv 1/k_BT$, and
$\Delta\varepsilon_R$ is the binding energy of a repressor to its operator site.
The derivation of this result is deferred to Appendix~\ref{sec:non_bursty}.

The fold-change is  found as the ratio of mean mRNA with and without repressor
as introduced in eqn.~\ref{eq:fc_def}. Invoking that definition results in
\begin{equation}
\text{fold-change}
= \left(1 + \frac{R}{N_{NS}} e^{-\beta\Delta\varepsilon_R}\right)^{-1},
\end{equation}
which clearly matches the form of the master curve in
Figure~\ref{fig1:means_cartoons}(C) with $\rho=1$ and $\Delta F_R = 
\beta\Delta\varepsilon_r - \log (R / N_{NS})$.

In fact it was noted in~\cite{Chure2019} that this two-state model can be viewed
as the coarse-graining of any equilibrium promoter model in which no
transcriptionally active states have transcription factor bound. Put
differently, there is no overlap between TF bound states and transcriptionally
active states. We will see this explicitly in the 3-state equilibrium model
below, but perhaps surprising is that an analogous result carries over even to
the nonequilibrium models we consider later.

\subsubsection{The Three-State Equilibrium Model}
Compared to the previous model, here we fine-grain the repressor unbound state
into two separate states: empty, and RNAP bound as shown in (2) in
Figure~\ref{fig1:means_cartoons}(B). This picture was used in~\cite{Garcia2011a}
as we use it here, and in~\cite{Razo-Mejia2018} and~\cite{Chure2019} it was
generalized to incorporate small-molecule inducers that bind the repressor. 
% Interestingly, the effect of this generalization is
% simply to renormalize $R$ from the total number of repressors to a smaller
% effective number, roughly (though not exactly) the number which are unbound by
% inducers.

The mean mRNA, as derived in the appendix from a similar enumeration of states
and weights as the previous model, is
\begin{equation}
\langle m \rangle = \frac{r}{\gamma}
\frac{\frac{P}{N_{NS}} e^{-\beta\Delta\varepsilon_P}}
        {
        1 + \frac{R}{N_{NS}} e^{-\beta\Delta\varepsilon_R}
        + \frac{P}{N_{NS}} e^{-\beta\Delta\varepsilon_P}
        },
\end{equation}
where the new variables are $\Delta\varepsilon_P$, the difference in RNAP
binding energy to its specific site (the promoter) relative to an average
nonspecific background site, and the RNAP copy number, $P$. The fold-change
again follows immediately as
\begin{align}
\text{fold-change}
&= \frac{\frac{P}{N_{NS}} e^{-\beta\Delta\varepsilon_P}}
        {
        1 + \frac{R}{N_{NS}} e^{-\beta\Delta\varepsilon_R}
        + \frac{P}{N_{NS}} e^{-\beta\Delta\varepsilon_P}
        }
\frac{1 + \frac{P}{N_{NS}} e^{-\beta\Delta\varepsilon_P}}
        {\frac{P}{N_{NS}} e^{-\beta\Delta\varepsilon_P}}
\\
&= \left(
1 + \frac{\frac{R}{N_{NS}} e^{-\beta\Delta\varepsilon_R}}
        {1 + \frac{P}{N_{NS}} e^{-\beta\Delta\varepsilon_P}}
\right)^{-1}
\\
&= (1 + \exp(-\Delta F_R - \log\rho))^{-1},
\end{align}
with $\Delta F_R = \beta\Delta\varepsilon_R - \log(R/N_{NS})$ and $\rho = 1 +
\frac{P}{N_{NS}}\mathrm{e}^{-\beta\Delta\varepsilon_P}$ as shown
in~\fig{fig1:means_cartoons}(B). Thus far, we see that the two thermodynamic
models, despite making different coarse-graining commitments, result in the same
functional form for the fold-change in mean gene expression.  We now explore how
kinetic models fare when faced with computing the same observable.

\subsubsection{The Poisson Promoter Nonequilibrium Model}
For our first kinetic model, we  imitate the states considered in the Two-State
Equilibrium Model and consider the simplest possible picture with only two
states, repressor bound and unbound. This is exactly the model used for the main
results of~\cite{Jones2014}. In this picture, repressor association and
dissociation rates from its operator site, $k_R^+$ and $k_R^-$, respectively,
govern transitions between the two states. When the system is in the unbound
state, transcription initiates at rate $r$, which represents a coarse-graining
of all the downstream processes into a single effective rate. mRNA is degraded
at rate $\gamma$ as already exploited in the previous models.

Let $p_{m,R}(t)$ denote the joint probability of finding the system in the
repressor bound state $R$ with $m$ mRNA molecules present at time $t$. Similarly
define $p_{m,U}(t)$ for the repressor unbound state $U$. This model is governed
by coupled master equations giving the time evolution of $p_{m,R}(t)$ and
$p_{m,U}(t)$~\cite{Sanchez2008, Sanchez2011, Phillips2019} which we can write as
\begin{align}
\begin{split}
\deriv{t}p_{m,R}(t) =& 
- \overbrace{k_R^- p_{m,R}(t)}^{R \rightarrow U}
+ \overbrace{k_R^+ p_{m,U}(t)}^{U \rightarrow R}
+ \overbrace{(m+1)\gamma p_{m+1,R}(t)}^{m + 1 \rightarrow m}
- \overbrace{\gamma p_{m,R}(t)}^{m \rightarrow m - 1}
\\
\deriv{t}p_{m,U}(t) =&\; 
\overbrace{k_R^- p_{m,R}(t)}^{R \rightarrow U}
- \overbrace{k_R^+ p_{m,U}(t)}^{U \rightarrow R}
+ \overbrace{rp_{m-1,U}(t)}^{m-1 \rightarrow m}
- \overbrace{rp_{m,U}(t)}^{m \rightarrow m + 1}
\\
&+ \overbrace{(m+1)\gamma p_{m+1,U}(t)}^{m + 1 \rightarrow m}
- \overbrace{\gamma p_{m,U}(t)}^{m \rightarrow m - 1},
\label{eq:poisson_promoter_cme}
\end{split}
\end{align}
where each term on the right corresponds to a transition between two states of
the promoter as indicated by the overbrace label. In each equation, the first
two terms describe transitions between promoter states due to repressors
unbinding and binding, respectively. The final two terms describe degradation of
mRNA, decreasing the copy number by one, and the terms with coefficient $r$
describe transcription initiation increasing the mRNA copy number by one.

We can greatly simplify the notation, which will be especially useful for the
more complicated models yet to come, by re-expressing the master equation in
vector form~\cite{Phillips2012}. The promoter states are collected into a vector
and the rate constants are collected into matrices as
\begin{equation}
\vect{p}_m = \begin{pmatrix} p_{m,R} \\ p_{m,U} \end{pmatrix},\
\matr{K} = \begin{pmatrix} -k_R^- & k_R^+ \\ k_R^- & -k_R^+ \end{pmatrix},\
\matr{R} = \begin{pmatrix} 0 & 0 \\ 0 & r \end{pmatrix},\
\label{eq:2state_cme_matrices}
\end{equation}
so that the master equation may be condensed as
\begin{equation}
\deriv{t}\vect{p}_m(t) =
\left( \matr{K} - \matr{R} - \gamma m \matr{I} \right) \vect{p}_m(t)
                + \matr{R} \vect{p}_{m-1}(t)
                + \gamma (m+1) \matr{I} \vect{p}_{m+1}(t),
\label{eq:2state_rep_cme}
\end{equation}
where $\matr{I}$ is the identity matrix. Taking steady state by setting time
derivatives to zero, the mean mRNA can be found to be
\begin{equation}
\langle m \rangle = \frac{r}{\gamma}
        \left(1 + \frac{k_R^+}{k_R^-}\right)^{-1},
\end{equation}
with the algebra details again deferred to Appendix~\ref{sec:non_bursty}. Recall
$k_R^+$ is proportional to the repressor copy number, so in computing
fold-change, absence of repressor corresponds to $k_R^+\rightarrow0$. Therefore
fold-change in this model is simply
\begin{equation}
\text{fold-change} = \left(1 + \frac{k_R^+}{k_R^-}\right)^{-1},
\end{equation}
again matching the master curve of~\fig{fig1:means_cartoons}(D) with $\rho=1$.

\subsubsection{Nonequilibrium Model Two - RNAP Bound and Unbound States}
Our second kinetic model depicted in Figure~\ref{fig1:means_cartoons}(C) mirrors
the second equilibrium model of Figure~\ref{fig1:means_cartoons}(B) by
fine-graining  the repressor unbound state of nonequilibrium model 1, resolving
it into an empty promoter state and an RNAP-bound state. Note in this picture,
in contrast with model 4 below, transcription initiation is accompanied by a
promoter state change, in keeping with the interpretation as RNAP-bound and
empty states: if an RNAP successfully escapes the promoter and proceeds to
elongation of a transcript, clearly it is no longer bound at the promoter.
Therefore another RNAP must bind before another transcript can be initiated.

The master equation governing this model is analogous
to~\eqrange{eq:2state_cme_matrices}{eq:2state_rep_cme} for model 1 above. The
main subtlety arises since transcription initiation accompanies a promoter state
change. This can be understood by analogy to $\matr{K}$. The off-diagonal and
diagonal elements of $\matr{K}$ correspond to transitions arriving at or
departing from, respectively, the promoter state of interest. If transcription
initiation is accompanied by promoter state changes, we must have separate
matrices for arriving and departing transcription events since the arriving and
departing transitions have different initial copy numbers of mRNA, unlike for
$\matr{K}$ where they are the same. The master equation for this model is
\begin{equation}
\deriv{t}\vect{p}_m(t) =
\left( \matr{K} - \matr{R_D} - \gamma m \matr{I} \right) \vect{p}_m(t)
                + \matr{R_A} \vect{p}_{m-1}(t) +
                \gamma (m+1) \matr{I} \vect{p}_{m+1}(t),
\label{eq:3state_rep_cme}
\end{equation}
with the state vector and promoter transition matrix defined as
\begin{equation}
\vect{p}_m = \begin{pmatrix} p_{m,R} \\ p_{m,E} \\ p_{m,P} \end{pmatrix},\
\matr{K} = \begin{pmatrix} -k_R^- & k_R^+ & 0 \\
                        k_R^- & -k_R^+ -k_P^+ & k_P^- \\
                        0 & k_P^+ & -k_P^- 
                \end{pmatrix},
\label{eq:3state_cme_matrices_pt1}
\end{equation}
and the initiation matrices given by
\begin{equation}
\matr{R_A} = \begin{pmatrix}
                0 & 0 & 0 \\ 
                0 & 0 & r \\ 
                0 & 0 & 0
                \end{pmatrix},\
\matr{R_D} = \begin{pmatrix}
                0 & 0 & 0 \\ 
                0 & 0 & 0 \\ 
                0 & 0 & r
                \end{pmatrix}.
\label{eq:3state_cme_matrices_pt2}
\end{equation}
The elements of $\vect{p}_m$ encode the probabilities of having $m$ mRNA present
along with the promoter having repressor bound ($R$), being empty ($E$), or
having RNAP bound ($P$), respectively. $\matr{R_A}$ describes probability flux
\textit{arriving} at the state $\vect{p}_m$ from a state with one fewer mRNA,
namely $\vect{p}_{m-1}$, and $\matr{R_D}$ describes probability flux
\textit{departing} from the state $\vect{p}_m$ for a state with one more mRNA,
namely $\vect{p}_{m+1}$. $\matr{K}$ is closely analogous to model 1.

Mean mRNA at steady state is found analogously to model 1, with the result
\begin{equation}
\langle m\rangle = \frac{r}{\gamma}
        \frac{k_R^- k_P^+}
        {k_R^- k_P^+ + k_R^- (k_P^- + r) + k_R^+ (k_P^- + r)}.
\end{equation}
Fold-change is again found from the ratio prescribed by~\eq{eq:fc_def}, from
which we have
\begin{align}
\text{fold-change}
&=      \frac{k_R^- k_P^+}
        {k_R^- k_P^+ + k_R^- (k_P^- + r) + k_R^+ (k_P^- + r)}
        \frac{k_P^+ + k_P^- + r}{k_P^+}
\\
&=      \left(1 + \frac{k_R^+}{k_R^-}
                \frac{k_P^- + r}{k_P^+ + k_P^- + r}
        \right)^{-1}
\\
&=      \left(1 + \frac{k_R^+}{k_R^-}
        \left(1 + \frac{k_P^+}{k_P^- + r}\right)^{-1}
        \right)^{-1},
\end{align}
which follows the master curve of~\fig{fig1:means_cartoons}(D) with $\rho = 1 +
k_P^+/(k_P^- + r)$ as claimed.

\subsubsection{Nonequilibrium Model Three - Multistep Transcription Initiation
and Escape}
One might reasonably complain that the first two ``nonequilibrium'' models we
have considered are straw men. Their steady states necessarily satisfy detailed
balance which is equivalent to thermodynamic equilibrium. Why is this the case?
At steady state there is by definition no net probability flux in or out of each
promoter state, but since the promoter states form a linear chain, there is only
one way in or out of the repressor bound and RNAP bound states, implying each
edge must actually have a net zero probability flux, which is the definition of
detailed balance (usually phrased as equality of forward and reverse transition
fluxes).

Now we consider model 3 in Figure~\ref{fig1:means_cartoons}(C) which allows the
possibility of true nonequilibrium steady-state fluxes through the promoter
states. We point out that this model was considered previously
in~\cite{Mitarai2015} where a comparison was made with model 1 as used
in~\cite{Jones2014}. The authors of~\cite{Mitarai2015} argued that the
additional complexity is essential to properly account for the noise in the mRNA
distribution. We will weigh in on both models later when we consider observables
beyond fold-change.

The master equation governing this model is identical in form to model 2 above,
namely
\begin{equation}
\deriv{t}\vect{p}_m(t) =
\left( \matr{K} - \matr{R_D} - \gamma m \matr{I} \right) \vect{p}_m(t)
                + \matr{R_A} \vect{p}_{m-1}(t) +
                \gamma (m+1) \matr{I} \vect{p}_{m+1}(t),
\end{equation}
but obviously with a higher-dimensional state space and different matrices. The
state vector and promoter transition matrix are now
\begin{equation}
\vect{p}_m = \begin{pmatrix} p_{m,R} \\ p_{m,E} \\
                             p_{m,C} \\ p_{m,O}\end{pmatrix},\
\matr{K} = \begin{pmatrix} -k_R^- & k_R^+ & 0 & 0\\
                        k_R^- & -k_R^+ -k_P^+ & k_P^- & 0 \\
                        0 & k_P^+ & -k_P^- - k_O & 0 \\
                        0 & 0 & k_O & 0
                \end{pmatrix},
\end{equation}
with the four promoter states, in order, being repressor bound ($R$), empty
($E$), RNAP closed complex ($C$), and RNAP open complex ($O$). Besides
increasing dimension by one, the only new feature in $\matr{K}$ is the
rate $k_O$, representing the (irreversible) rate of open complex formation from
the closed complex. The initiation matrices are given by
\begin{equation}
\matr{R_A} = \begin{pmatrix}
        0 & 0 & 0 & 0 \\ 
        0 & 0 & 0 & r \\ 
        0 & 0 & 0 & 0 \\ 
        0 & 0 & 0 & 0
                \end{pmatrix},\
\matr{R_D} = \begin{pmatrix}
        0 & 0 & 0 & 0 \\ 
        0 & 0 & 0 & 0 \\ 
        0 & 0 & 0 & 0 \\ 
        0 & 0 & 0 & r
                \end{pmatrix},
\end{equation}
again closely analogous to nonequilibrium model 2.

The expression for mean mRNA is substantially more complicated now, as worked
out in Appendix~\ref{sec:non_bursty} where we find
\begin{equation}
\langle m\rangle = \frac{r}{\gamma}
        \frac{k_R^- k_P^+ k_O}
        {k_R^- [(k_P^+ (k_O + r) + r(k_P^- + k_O)] + k_R^+ r(k_P^- + k_O)},
\end{equation}
which can be simplified to
\begin{equation}
\langle m\rangle
= \frac{r}{\gamma}
\frac{\frac{k_P^+ k_O}{r(k_O + k_P^-)}}
        {1 + \frac{k_P^+ (k_O + r)}{r(k_O + k_P^-)} + \frac{k_R^+}{k_R^-}}.
\end{equation}
The strategy is to isolate the terms involving the repressor, so that now the
fold-change is seen to be simply
\begin{align}
\text{fold-change}
&= \frac{\frac{k_P^+ k_O}{r(k_O + k_P^-)}}
        {1 + \frac{k_P^+ (k_O + r)}{r(k_O + k_P^-)} + \frac{k_R^+}{k_R^-}}
        \frac{1 + \frac{k_P^+ (k_O + r)}{r(k_O + k_P^-)}}
                {\frac{k_P^+ k_O}{r(k_O + k_P^-)}}
\\
&= \left(
        1 + \frac{k_R^+}{k_R^-}
        \left(1 + \frac{k_P^+ (k_O + r)}{r(k_O + k_P^-)}\right)^{-1}
\right)^{-1},
\end{align}
surprisingly reducing to the master curve of~\fig{fig1:means_cartoons}(D) once
again, with $\rho = 1 + \frac{k_P^+ (k_O + r)}{r(k_O + k_P^-)}$.

This example hints that an arbitrarily fine-grained model of downstream
transcription steps may still be collapsed to the form of the master curve for
the means giving in~\fig{fig1:means_cartoons}(D) so long as the repressor
binding is exclusive with transcriptionally active states. We offer this as a
conjecture, and we suspect that a careful argument using the King-Altman diagram
method~\cite{King1956, Hill1966} might furnish a ``proof.'' Our focus here is
not on full generality but rather to survey an assortment of plausible models
for simple repression  that have been proposed in the literature.

\subsubsection{Nonequilibrium Model Four - ``Active'' and ``Inactive'' States}
This model, shown as (4) in Figure~\ref{fig1:means_cartoons}(C) is at the core
of the theory in~\cite{Razo-Mejia2020}. At a glance the cartoon for this model
may appear very similar to model 2, and mathematically it is, but the
interpretation is rather different. In model 2, we interpreted the third state
literally as an RNAP-bound promoter and modeled initiation of a transcript as
triggering a promoter state change, making the hopefully uncontroversial
assumption that an RNAP can only make one transcript at a time. In contrast, in
the present model the promoter state does \textit{not} change when a transcript
is initiated. So we no longer interpret these states as literally RNAP bound and
unbound but instead as coarse-grained ``active'' and ``inactive'' states, the
details of which we leave unspecified for now. We will comment more on this
model below when we discuss Fano factors of models.

Mathematically this model is very similar to models 1 and 2. Like model 1, the
matrix $R$ describing transcription initiation is diagonal, namely
\begin{equation}
\matr{R} = \begin{pmatrix}
                0 & 0 & 0 \\ 
                0 & 0 & 0 \\ 
                0 & 0 & r
        \end{pmatrix}.
\end{equation}
The master equation takes verbatim the same form as it did for model
1,~\eq{eq:2state_rep_cme}. Meanwhile the promoter transition matrix $K$ is the
same as~\eq{eq:3state_cme_matrices_pt1} from model 2, although we relabel the
rate constants from $k_P^\pm$ to $k^\pm$ to reiterate that these are not simply
RNAP binding and unbinding rates.

Carrying out the algebra, the mean mRNA can be found to be
\begin{equation}
\langle m\rangle = \frac{r}{\gamma}
\frac{k_R^- k^+}
{k_R^- k^+ + k_R^- k^- + k_R^+ k^-},
\end{equation}
and the fold-change readily follows,
\begin{align}
\text{fold-change}
&=      \frac{k_R^- k^+}{k_R^- k^+ + k_R^- k^- + k_R^+ k^-}
        \frac{k_R^- k^+ + k_R^- k^-}{k_R^- k^+}
\\
&=      \left(1 + \frac{k_R^+}{k_R^-}
                \left(1 + \frac{k^+}{k^-}\right)^{-1}
        \right)^{-1},
\end{align}
from which we see $\rho = 1 + k^+/k^-$ as shown in~\fig{fig1:means_cartoons}(C).

\subsubsection{Nonequilibrium Model Five - Bursty Promoter}
The final model we consider shown in Figure~\ref{fig1:means_cartoons}(C) is an
intuitive analog to model 1, with just two states, repressor bound or unbound,
and transition rates between them of $k_R^+$ and $k_R^-$. In model 1, when in
the unbound state, single mRNA transcripts are produced as a Poisson process
with some characteristic rate $r$. The current model by contrast produces, at
some Poisson rate $k_i$, \textit{bursts} of mRNA transcripts with a
geometrically distributed burst size and a mean burst size $b$.

From this intuitive picture and by analogy to model 1, then, it should be
plausible that the mean mRNA level is
\begin{equation}
\langle m\rangle = \frac{k_i b}{\gamma}
        \left(1 + \frac{k_R^+}{k_R^-}\right)^{-1},
\end{equation}
which will turn out to be correct from a careful calculation. For now, we simply
note that just like model 1, the fold-change becomes
\begin{equation}
\text{fold-change} = \left(1 + \frac{k_R^+}{k_R^-}\right)^{-1}
\end{equation}
with $\rho=1$ also like model 1.
We will also see later how this model emerges as a natural limit of model 4.

\subsubsection{Summary of Results across models for Fold-Changes in Mean
Expression}
What have we learned from our exploration of the models
in~\fig{fig1:means_cartoons} and their predictions of fold-changes in mean
expression levels? The key outcome of our analysis is the existence of a master
curve shown in \fig{fig1:means_cartoons}(D) to which the fold-change predictions
of all the models collapse. This master curve is parametrized by only two
effective parameters: $\Delta F_R$, which characterizes the number of repressors
and their binding strength to the DNA, and $\rho$, which characterizes all other
features of the promoter architecture. The key assumption underpinning this
result is that no transcription occurs when a repressor is bound to its
operator. Note, however, that we are agnostic about the molecular mechanism
which achieves this; steric effects are one plausible mechanism, but, for
instance, ``action-at-a-distance'' mediated by kinked DNA due to repressors
bound tens or hundreds of nucleotides upstream of a promoter is plausible as
well.

Why does the master curve of~\fig{fig1:means_cartoons}(D) exist at all? This
brings to mind the deep questions posed in, e.g.,~\cite{Frank2013}
and~\cite{Frank2014a}, asking us to search for general ``symmetries'' of the
problem instead of specifying generative models of the known microscopics of a
system. In our case, the key feature seems to be the exclusive nature of
repressor and RNAP binding, which allows the parameter describing the repressor,
$\Delta F_R$, to cleanly separate from all other details of the promoter
architecture, which are encapsulated in $\rho$. In the graph-theoretic language
which was applied to similar problems in~\cite{Gunawardena2012}
and~\cite{Ahsendorf2014}, the repressor-bound state is a leaf of the graph
describing transitions between promoter states. Arbitrary nonequilibrium
behavior can occur on the rest of the promoter state space, but it may all be
swept up in the effective parameter $\rho$, to which the repressor makes no
contribution.

As suggested in~\cite{Chure2019}, we believe this master curve should generalize
to architectures with multiple repressor binding sites, as long as the
exclusivity of transcription factor binding and transcription initiation is
maintained. The interpretation of $\Delta F_R$ is then of an effective free
energy of all repressor bound states. In an equilibrium picture this is simply
given by the log of the sum of Boltzmann weights of all repressor bound states,
which looks like the log of a partition function of a subsystem. In a
nonequilibrium picture, while we can still mathematically gather terms and give
the resulting collection the label $\Delta F_R$, it is unclear if the physical
interpretation as an effective free energy makes sense. As is well known, free
energies cannot be assigned unambiguously to states out of equilibrium because
the free energy change along a generic path traversing the state space is path
dependent, unlike at equilibrium. A consequence of this is that, out of
equilibrium, $\Delta F_R$ is no longer a simple sum of Boltzmann weights.
Instead it resembles a restricted sum of King-Altman diagrams~\cite{King1956,
Hill1966}. Following the work of Hill~\cite{Hill1989}, it may yet be possible to
interpret this expression as an effective free energy, but this remains unclear
to us. We leave this an open problem for future work.

If we relax the requirement of exclusive repressor-RNAP binding, one could
imagine models in which repressor and RNAP doubly-bound states are allowed,
where the repressor's effect is to \text{reduce} the transcription rate rather
than setting it to zero. Our results do not strictly apply to such a model,
although we note that if the repressor's reduction of the transcription rate is
substantial, such a model might still be well-approximated by one of the models
in~\fig{fig1:means_cartoons}.

One may worry that our ``one curve to rule them all'' is a mathematical
tautology. In fact we \textit{agree} with this criticism if $\Delta F_R$ is
``just a fitting parameter'' and cannot be meaningfully interpreted as a real,
physical free energy. An analogy to Hill functions is appropriate here. One of
their great strengths and weaknesses, depending on the use they are put to, is
that their parameters coarse-grain many details and are generally not
interpretable in terms of microscopic models, for deep reasons discussed at
length in~\cite{Frank2013}. By contrast, our master curve claims to have the
best of both worlds: a coarse-graining of all details besides the repressor into
a single effective parameter $\rho$, while simultaneously retaining an
interpretation of $\Delta F_R$ as a physically meaningful and interpretable free
energy. Our task, then, is to prove or disprove this claim.

How do we test this and probe the theory with fold-change measurements? There is
a fundamental limitation in that the master curve is essentially a one-parameter
function of $\Delta F_R + \log\rho$. Worse, there are many \textit{a priori}
plausible microscopic mechanisms that could contribute to the value of $\rho$,
such as RNAP binding and escape kinetics~\cite{DeHaseth1998, Mitarai2015},
and/or supercoiling accumulation and release~\cite{Chong2014, Sevier2016},
and/or, RNAP clusters analogous to those obvserved in
eukaryotes~\cite{Cisse2013, Cho2016} and recently also observed in
bacteria~\cite{Ladouceur2020}. Even if $\Delta F_R$ is measured to high
precision, inferring the potential microscopic contributions to $\rho$, buried
inside a log no less, from fold-change measurements seems beyond reach. As a
statistical inference problem it is entirely nonidentifiable, in the language
of~\cite{Gelman2013}, Section 4.3.

If we cannot simply infer values of $\rho$ from measurements of fold-change, can
we perturb some of the parameters that make up $\rho$ and measure the change?
Unfortunately we suspect this is off-limits experimentally: most of the
potential contributors to $\rho$ are global processes that affect many or all
genes. For instance, changing RNAP association rates by changing RNAP copy
numbers, or changing supercoiling kinetics by changing topoisomerase copy
numbers, would massively perturb the entire cell's physiology and confound any
determination of $\rho$.

One might instead imagine a bottom-up modeling approach, where we mathematicize
a model of what we hypothesize the important steps are and are not, use in vitro
data for the steps deemed important, and \textit{predict} what $\rho$ should be.
But again, because of the one-parameter nature of the master curve, many
different models will likely make indistinguishable predictions, and without any
way to experimentally perturb in vivo, there is no clear way to test whether the
modeling assumptions are correct.

In light of this, we prefer the view that parameters and rates are not directly
comparable between cartoons in~\fig{fig1:means_cartoons}. Rather, parameters in
the simpler cartoons represent coarse-grained combinations of parameters in the
finer-grained models. For instance, by equating $\rho$ between any two models,
one can derive various possible correspondences between the two models'
parameters. Note that these correspondences are clearly not unique, since many
possible associations could be made. It then is a choice as to what microscopic
interpretations the model-builder prefers for the parameters in a particular
cartoon, and as to which coarse-grainings lend intuition and which seem
nonsensical. Indeed, since it remains an open question what microscopic features
dominate $\rho$ (as suggested above, perhaps RNAP binding and escape
kinetics~\cite{DeHaseth1998, Mitarai2015}, or supercoiling accumulation and
release~\cite{Chong2014, Sevier2016}, or, something more exotic like RNAP
clusters~\cite{Cisse2013, Cho2016, Ladouceur2020}), we are hesitant to put too
much weight on any one microscopic interpretation of model parameters that make
up $\rho$.

One possible tuning knob to probe $\rho$ that would not globally perturb the
cell's physiology is to manipulate RNAP binding sites. Work such
as~\cite{Kinney2010} has shown that models of sequence-dependent RNAP can be
inferred from data, and the authors of~\cite{Brewster2012} showed that the model
of~\cite{Kinney2010} has predictive power by using the model to \textit{design}
binding sites of a desired affinity. But for our purposes, this begs the
question: the authors of~\cite{Kinney2010} \textit{assumed} a particular model
(essentially our 3-state equilibrium model but without the repressor), so it is
unclear how or if such methods can be turned around to \textit{compare}
different models of promoter function.

We have already pointed out that the master curve of~\fig{fig1:means_cartoons}
is essentially a one-parameter model, the one parameter being $\Delta F_R +
\log\rho$. By now the reader may be alarmed as to how can we even determine
$\Delta F_R$ and $\rho$ independently of each other, never mind shedding a lens
on the internal structure of $\rho$ itself. A hint is provided by the weak
promoter approximation, invoked repeatedly in prior studies~\cite{Bintu2005c,
Garcia2011a, Razo-Mejia2018} of simple repression using the 3-state equilibrium
model in~\fig{fig1:means_cartoons}(B). In that picture, the weak promoter
approximation means $\frac{P}{N_{NS}}\exp(-\beta\Delta\varepsilon_P) \ll 1$,
meaning therefore $\rho\approx1$.  This approximation can be well justified on
the basis of the number of RNAP per cell and the strength of binding of RNAP to
DNA at weak promoters. This is suggestive, but how can we be sure that $\rho$
is not, for instance, actually $10^2$ and that $\Delta F_R$ hides a compensatory
factor? A resolution is offered by an independent inference of $\rho$ in the
absence of repressors. This was done in~\cite{Razo-Mejia2020} by fitting
nonequilibrium model 4 in~\fig{fig1:means_cartoons}(C), with zero repressor
(looking ahead, this is equivalent to model 4 in~\fig{fig2:constit_cartoons}), to
FISH data from~\cite{Brewster2014}. This provided a determination of $k^+$ and
$k^-$, from which their ratio is estimated to be no more than a few $10^{-1}$
and possibly as small as $10^{-2}$.

The realization that $\rho\approx1$ to an excellent approximation,
\textit{independent} of which model in~\fig{fig1:means_cartoons} one prefers,
goes a long way towards explaining the surprising success of equilibrium models
of simple repression. Even though our 2- and 3-state models get so many details
of transcription wrong, it does not matter because fold-change is a cleverly
designed ratio. Since $\rho$ subsumes all details except the repressor, and
$\log\rho\approx0$, fitting these simple models to fold-change measurements can
still give a surprisingly good estimate of repressor binding energies. So the
ratiometric construction of fold-change fulfills its intended purpose of
canceling out all features of the promoter architecture except the repressor
itself. Nevertheless it is perhaps surprising how effectively it does so:
\textit{a priori}, one might not have expected $\rho$ to be quite so close to 1.

We would also like to highlight the relevance of~\cite{Landman2019} here.
Landman et.\ al.\ reanalyzed and compared in vivo and in vitro data on the lacI
repressor's binding affinity to its various operator sequences. (The in vivo
data was from, essentially, fitting our master curve to expression
measurements.) They find broad agreement between the in vitro and in vivo
values. This reinforces the suspicion that the equilibrium $\Delta\varepsilon_R$
repressor binding energies do in fact represent real physical free energies.
Again, \textit{a priori} this did not have to be the case, even knowing that
$\rho\approx1$.

In principle, if $\Delta F_R$ can be measured to sufficient precision, then
deviations from $\rho=1$ become a testable matter of experiment. In practice, it
is probably unrealistic to measure repressor rates $k_R^+$ or $k_R^-$ or
fold-changes in expression levels (and hence $\Delta\varepsilon_R$) precisely
enough to detect the expected tiny deviations from $\rho=1$. We can estimate the
requisite precision in $\Delta F_R$ to resolve a given $\Delta\rho$ by noting,
since $\rho\approx1$, that $\log(1+\Delta\rho)\approx \Delta\rho$, so
$\Delta(\Delta F_R) \approx \Delta\rho$. Suppose we are lucky and $\Delta\rho$
happens to be $\sim0.1$, on the high end of our range estimated above. A
determination of $\Delta\varepsilon_R/k_BT$ with an uncertainty of barely 0.1
was achieved in the excellent measurements of~\cite{Razo-Mejia2018}, so this
requires a very difficult determination of $\Delta F_R$ for a very crude
determination of $\rho$, which suggests, to put it lightly, this is not a
promising path to pursue experimentally. It is doubtful that inference of
repressor kinetic rates would be any easier.\footnote{\mmnote{In fact the
uncertainties quoted in Hammar et.\ al.~\cite{Hammar2014} from Elf's group are
small enough to do this calculation, but HJ \& I think their error bars are
insanely optimistic. Not sure I want to blatantly call them out on it though??}}

Moving forward, we have weak evidence supporting the interpretation of $\Delta
F_R$ as a physically real free energy~\cite{Landman2019} and other work casting
doubt~\cite{Hammar2014}. How might we resolve the confusion? If there is no
discriminatory power to test the theory and distinguish the various models with
measurements of fold-changes in means, how do we probe the theory? Clearly to
discriminate between the nonequilibrium models in~\fig{fig1:means_cartoons}, we
need to go beyond means to ask questions about kinetics, noise and even full
distributions of mRNA copy numbers over a population of cells. If the
``one-curve-to-rule-them-all'' is more than a mathematical tautology, then the
free energy of repressor binding inferred from fold-change measurements should
agree with repressor binding and unbinding rates. In other words, the
equilibrium and nonequilibrium definitions of $\Delta F_R$ must agree, meaning
\begin{equation}
\Delta F_R = \beta\Delta\varepsilon_R - \log(R/N_{NS})
        = - \log(k_R^+/k_R^-),
\label{eq:deltaFR_eq_noneq_equiv}
\end{equation}
must hold, where $\beta\Delta\varepsilon_R$ is inferred from the master curve
fit to fold-change measurements, and $k_R^+$ and $k_R^-$ are inferred in some
orthogonal manner. Single molecule measurements such as from~\cite{Hammar2014}
have directly observed these rates, and in the remainder of this work we explore
a complementary approach: inferring repressor binding and unbinding rates
$k_R^+$ and $k_R^-$ from FISH measurements of mRNA population distributions.
\marginpar{\it MM: Add a notation aside somewhere that all association rates are written as zeroth-order rates. In other words, the concentration of the molecule is hidden inside, so for instance, $k_R^+$ is proportional to repressor copy number.}