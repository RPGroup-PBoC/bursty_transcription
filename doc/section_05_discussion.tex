% !TEX root = ./busty_transcription.tex
\section{Discussion and future work}

The study of gene expression is one of the dominant themes of modern biology,
made all the more urgent by the dizzying pace at which genomes are being
sequenced. But there is a troubling Achilles heel buried in all of that genomic
data and that is our inability to find and interpret regulatory sequence. In
many cases, this is not possible even qualitatively, let alone the possibility
of quantitative dissection of the regulatory parts of genomes in a predictive
fashion. Other recent work has tackled the challenge of finding and annotating
the regulatory part of genomes \cite{Belliveau2018, Ireland2020} \mrm{add other
non-self referential}. Once we have determined the architecture of the
regulatory part of the genome, we are then faced with the next class of
questions which are sharpened by formulating them in mathematical terms, namely,
what are the input-output properties of these regulatory circuits and what knobs
control them?

The present work has tackled that question in the context of the first
regulatory architecture hypothesized in the molecular biology era, namely, the
repressor-operator model of Jacob and Monod. Regulation in that architecture is
the result of a competition between a repressor which inhibits transcription and
RNAP polymerase which undertakes it. Through the labors of generations of
geneticists, molecular biologists and biochemists, an overwhelming amount of
information and insight has been garnered into this simple regulatory motif,
licensing it as what one might call the ``hydrogen atom'' of regulatory biology.
It is from that perspective that the present paper explores the extent to which
some of the different models that have been articulated to describe that motif
allow us to understand both the average level of gene expression found in a
population of cells, the intrinsic cell-to-cell variability, and the full gene
expression distribution found in such population as would be reported in a FISH
experiment, for example.

Our key insights can be summarized as follows. First, as shown in 
Figure~\ref{fig1:means_cartoons}, the mean expression in the simple repression
architecture is captured by a master curve in which the action of repressor and
the details of the polymerase interaction with the promoter appear separately
and additively in an effective free energy. Interestingly, as has been shown
elsewhere in the context of the Monod-Wyman-Changeux model, these kinds of
coarse-graining results are an exact mathematical result and do not constitute
hopeful approximations or biological naivete \cite{Razo-Mejia2018, Chure2019}.
To further dissect the relative merits of the different models, we must appeal
to higher moments of the gene expression probability distribution. To that end,
our second set of insights focus on gene expression noise, where it is seen that
a treatment of the constitutive promoter already reveals that some models have
Fano factors (standard deviation/mean) that are less than one, at odds with
any and all experimental data that we are aware of.
\mrm{
This theoretical result allows us to directly discard a subset of the models
(models 1-3 in Figure~\ref{fig2:fig2:constit_cartoons}) since they cannot be
reconciled with experimental observations. The two models left (models 4 and 5
in Figure~\ref{fig2:fig2:constit_cartoons}) seem to provide a rich enough
phenomenology to be able to reproduce the data. A previous exploration of one 
of these models demonstrated the sloppy nature of the model in which single-cell
mRNA counts data alone cannot constrain the value of all parameters 
simultaneously \cite{Razo-Mejia2020}. Here we demonstrate that the proposed
one-state bursty promoter model (model 5 in 
Figure~\ref{fig2:fig2:constit_cartoons}) is a limit of the commonly used
two-state promoter model \cite{Peccoud1995}. We put at test the idea that this
level of coarse-graining is descriptive enough to reproduce previous 
experimental observations. In particular we perform Bayesian inference on the
two parameters describing the full steady-state mRNA distribution, finding that
the model is able to capture a plethora of promoter sequences with different
mean levels of expression and noise.
}

\mrm{
Here we should include a discussion of the regulated case.
}

It is interesting to speculate what microscopic details are being coarse-grained
by our burst rate and burst size in Figure~\ref{fig2:constit_cartoons} model 5.
Chromosomal locus is one possible influence we have not addressed in this work,
as all the FISH data from~\cite{Jones2014} that we considered was from a
construct integrated at the \textit{galK} locus. The results of~\cite{Chong2014}
indicate that transcription-induced supercoiling contributes substantially in
driving transcriptional bursting, and furthermore, their Figure~7 suggests that
the parameters describing the rate, duration, and size of bursts vary
substantially for transcription from different genomic loci. Although the
authors of~\cite{Englaender2017} do not address noise, they note enormous
differences in mean expression levels when an identical construct is integrated
at different genomic loci. The authors of~\cite{Engl2020} attribute noise and
burstiness in their FISH data to the influence of different sigma factors, which
is a reasonable conclusion from their data. Could the difference also be due to
the different chromosomal locations of the two operons?

It seems clear therefore that chromosomal locus plays a major role in mean
expression levels as well as noise. What features of different loci are and are
not important? Sigma factors could certainly play a part, but it seems many
other mechanisms are at play: supercoiling, and more generally 3D organization
of the chromosome, read-through of neighboring genes, and surely some
fascinating unknown unknowns. Could our preferred coarse-grained model capture
the variability across different loci? If so, and we were to repeat the
parameter inference as done in this work, is there a simple theoretical model we
could build to understand the resulting parameters?

\mmnote{Do all my approximations work simply because galK happens to have a low
duty cycle of supercoiling, $\beta/\alpha$ in Chong2014 picture? A potentially
interesting test: put in the dozen or so Hernan constructs at some other locus,
preferably one of the ones Chong identified as having a large duty cycle (large
$\beta/\alpha$), i.e., the opposite limit of galK. Then just measure
fold-changes. You don't even need to do FISH, although that would be even
better. If their model is right, $\rho$ should be approximately set by the duty
cycle of supercoiling, NOT by RNAP kinetics.
\textit{This is a NICE prediction for future work.} \textit{cspE}
is claimed nonessential, prediction is you drop HG reporter constructs in there
and the inferred binding energies of all lacI operators shift the same amount,
by about $\ln(3)\sim 1kT$. Not huge, but detectable, although might be
confounding effects from moving to a new locus. With FISH measurements, $k^+$
and $k^-$ in Fig 2, model 4 should now be separately inferrable if Chong2014
measurements of $\beta/\alpha$ are correct. This would require an obnoxious
amount of cloning, but once you did that, the measurements would be same old
stuff that RP lab does.}



A direct \mmnote{if cloning intensive} future experiment would be
to design a payload with a promoter driving a FISH reporter,
integrate that payload at a variety of loci scattered around the genome,
and carry out FISH measurements of the resulting strains.
\mmnote{
Did Brewster already do this? I remember he integrated arrays of
competitor binding sites all over the genome, but did he put
lacUV5 + reporter scattered around?}
Obviously integration loci would need to be carefully chosen so
that one could plausibly hope the deletion and integration did
not disturb the cell's physiology too badly.