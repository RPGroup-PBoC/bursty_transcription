\documentclass[12pt]{article}%{amsart}
\usepackage[top = 1.0in, bottom = 1.0in, left = 1.0in, right = 1.0in]{geometry}
% See geometry.pdf to learn the layout options. There are lots.
%\geometry{letterpaper}               % ... or a4paper or a5paper or ...
%\geometry{landscape}             % Activate for for rotated page geometry
\usepackage[parfill]{parskip}    % Activate to begin paragraphs with an empty line rather than an indent
\usepackage{graphicx}
\usepackage{amssymb}
%\usepackage{epstopdf}
% \usepackage{caption}
\usepackage{amsmath}
% \usepackage{longtable}
% \usepackage{tabu}
% \usepackage{accents} %to get undertildes for vec & mat
\usepackage{color} % for colored text
% \usepackage[normalem]{ulem} % for editing: striking out text using 'sout' command

% \newcommand\mytablefigwidth{0.35\textwidth}

% allows to use .ai files directly w/o resaving as pdf
\DeclareGraphicsRule{.ai}{pdf}{.ai}{}

% Handy math macros!
\newcommand{\vect}[1]{\vec{#1}}
\newcommand{\matr}[1]{\mathbf{#1}}
\newcommand{\rate}[3]{{#1}_{#2}^{#3}}
\newcommand{\mmnote}[1]{\textcolor{cyan}{(MM:~#1)}}

% derivative macros. these are sneaky. the [{}] is an empty default 1st arg
% usage: provide 1 arg: \deriv{x} gives d/dx
% provide 2 args like so: \deriv[f]{x} gives df/dx
\newcommand{\deriv}[2][{}]{\frac{d #1}{d #2}}
\newcommand{\pderiv}[2][{}]{\frac{\partial #1}{\partial #2}}

%%%%% Referencing macros %%%%%
\newcommand{\fref}[1]{Figure~\ref{#1}}
\newcommand{\tref}[1]{Table~\ref{#1}}
\newcommand{\eref}[1]{Eq.~(\ref{#1})}
\newcommand{\erngref}[2]{Eq.~(\ref{#1}-\ref{#2})} % Equations
% \newcommand{\test}[1][]{%
% \ifthenelse{\isempty{#1}{}}{omitted}{#1}%
% }

%%%%%
\begin{document}

\title{Generating function solution for the bursty unregulated promoter}

\maketitle

\section{One-state promoter with bursts, mRNA only}

\subsection{From master equation to generating function}
Before tackling mRNA and protein together as in~\cite{Shahrezaei2008}, let's
just do mRNA (replicating Charlotte's notes; she says it's worked out
in~\cite{Paulsson2000}, but I don't see it, at least not in notation I can
comprehend). The master equation of interest is
\begin{align}
\deriv{t}p(m,t) = (m+1)\gamma p(m+1,t) - m\gamma p(m,t) - r p(m,t)
        + r \sum_{m^\prime=0}^m G_{m-m^\prime}(\theta) p(m^\prime,t),
\label{eq:1state_unreg_003}
\end{align}
where $G_{k}(\theta)$ is the geometric distribution defined as
\begin{align}
G_{k}(\theta) = \theta(1 - \theta)^k, \, k\in\{0,1,2,\dots\}.
\end{align}
With this convention, the mean burst size $\beta = (1-\theta)/\theta$.
$\gamma$ and $r$ are mRNA degradation rates and transcription burst
initiation rates, resp. The last term represents all ways the system could
end up with $m$ mRNAs, having started the burst with $m^\prime$. Define
$\lambda = r/\gamma$ and nondimensionalize time by $\gamma$, giving
\begin{align}
\deriv{t}p(m,t) = (m+1)p(m+1,t) - m p(m,t) - \lambda p(m,t)
        + \lambda \sum_{m^\prime=0}^m G_{m-m^\prime}(\theta) p(m^\prime,t),
\end{align}
The probability generating function is defined as
\begin{align}
F(z,t) = \sum_m z^m p(m,t).
\end{align}
Multiply both sides of the CME by $z^m$ and sum over all $m$ to get
\begin{align}
\pderiv[F]{t} = (1 - z) \pderiv[F]{z}
        + \left(\frac{\theta}{1-z(1-\theta)}-1\right)\lambda F.
\end{align}
(This is just like in Charlotte's notes; the tricky bit is to
recognize the need to reverse the double sum in the last term from
$\sum_{m=0}^\infty\sum_{m^\prime=0}^m$ to
$\sum_{m^\prime=0}^\infty\sum_{m=m^\prime}^\infty$.
The sum on $m$ is then an easy geometric series. The rest of the procedure
is just the standard reindexing tricks for sums over master equations.)
Changing variables to $\xi=1-\theta$ and simplifying gives
\begin{align}
\pderiv[F]{t} + (z - 1) \pderiv[F]{z} = \frac{(z-1)\xi}{1-z\xi}\lambda F.
\label{eq:1state_unreg_015}
\end{align}
\subsection{Stead-state}
At steady-state, the PDE reduces to the ODE
\begin{align}
\deriv[F]{z} = \frac{\xi}{1-z\xi}\lambda F,
\end{align}
which we can integrate as
\begin{align}
\int \frac{dF}{F} = \int \frac{\lambda\xi dz}{1-\xi z}.
\end{align}
The initial conditions for generating functions always confuse me,
especially since most authors play fast and loose and assume ``it's
trivial.'' The key fact: from the definition
$F(z,t) = \sum_m z^m p(m,t)$,
normalization requires that
$F(z=1^-,t) = \sum_m p(m,t) = 1$.\footnote{
Sometimes the generating function may be undefined \textit{at} $z=1$ but the
limit still holds. Also people tend to change variables from $z$ to other
things, so don't lose track of how this condition transforms.
}
Doing the integrals (and producing constant $c$) gives
\begin{align}
\ln F &= -\lambda \ln(1-\xi z) + c
\\
F &= \frac{c}{(1-\xi z)^\lambda}.
\end{align}
Only one choice for $c$ can satisfy initial conditions, producing
\begin{align}
F(z) = \left(\frac{1-\xi}{1-\xi z}\right)^\lambda
        = \left(\frac{\theta}{1 - z(1-\theta)}\right)^\lambda,
\end{align}
which is exactly the negative binomial's generating function, as expected.

\subsection{Time-dependent case}
Return to~\eref{eq:1state_unreg_015} and for convenience,
change variables to $v=z-1$, producing
\begin{align}
\frac{1}{v}\pderiv[F]{t} + \pderiv[F]{v} = \frac{\lambda\xi}{1-(1+v)\xi} F.
\end{align}
This can be solved with the method of characteristics. Parametrize the
characteristics by the new variable $s$. Initial conditions at $t=0$
correspond to $s_0$, $v_0$, and $F(v_0,t=0)$, where $s_0$ and $v_0$ are
constants to be found later. The characteristic equations are
\begin{align}
\deriv[t]{s} &= \frac{1}{v}
\\
\deriv[v]{s} &= 1
\\
\deriv[F]{s} &= \frac{\lambda\xi}{1-(1+v)\xi} F.
\end{align}
We could immediately write $v=s+c_v$ for some constant $c_v$ yet to be
found. Most authors drop the constant without justification. This works
because our set of ODEs only contains $ds$, not $s$ itself. So, in fact,
we need not even solve for $v$ in terms of $s$, since the second ODE
tells us $ds=dv$, we can immediately remove $s$ from the problem and we
are left with the ODEs
\begin{align}
\deriv[t]{v} &= \frac{1}{v}
\\
\deriv[F]{v} &= \frac{\lambda\xi}{1-(1+v)\xi} F.
\end{align}
The first is trivial. Using our initial conditions it gives
\begin{align}
v = v_0 e^t.
\end{align}
Set this aside and tackle the second ODE.
It still permits separation of variables which gives
\begin{align}
\int \frac{dF}{F} &= \int \frac{\lambda \xi dv}{1 - (1+v)\xi}
\\
\ln F &= -\lambda \ln(1-(1+v)\xi) + \text{constant}.
\end{align}
Initial conditions determine the constant, but this is subtle. It may be
tempting to transform back to $z$ and $t$, for which initial conditions
are more obvious, but $v$ is a more convenient variable since it is
parametrizes the characteristics. In other words, we can view $F$ as a
one variable function $F(v)$ or as a two variable function $F(z,t)$. It
is much easier to deal with before changing variables back, rather than
after. But, in terms of $v$, what are our initial conditions? We know
$t=0$ corresponds to some $v_0$ we have yet to determine, and this
corresponds to some $F(v_0)$ yet to be found. Any parametrization of the
constant that enforces this is acceptable. A clever idea is to break up
and reparametrize the constant as $c_1$ and $c_2$ and bury these in the
logs as
\begin{align}
\ln \frac{F(v)}{c_1} &= -\lambda \ln\frac{1-(1+v)\xi}{c_2},
\end{align}
where we are now explicitly showing the $v$ dependence of $F$ for clarity.
For this equality to remain true when $v=v_0$,
the only choice of $c_1$ and $c_2$ that works is
\begin{align}
\ln \frac{F(v)}{F(v_0)} &= -\lambda \ln\frac{1-(1+v)\xi}{1-(1+v_0)\xi}
\end{align}
which leads immediately to
\begin{align}
F(v) &= F(v_0) \left(\frac{1-(1+v_0)\xi}{1-(1+v)\xi}\right)^{\lambda}
\end{align}
Now we may transform variables back to $z$ and $t$.
Recall from earlier that $1+v=z$ and $v=v_0 e^t$, so $v_0 = (z-1)e^{-t}$.
These substitutions handle the term in parentheses. $F(v_0)$ is more subtle.
We must avoid the trap of thinking there exists some $z_0$ corresponding to $v_0$;
$v_0$ corresponds only to $t=0$. This is the magic of characteristics.
$F(v_0)$ corresponds to the initial condition on $F$, i.e., $F(z,t=0)$
\textit{not} $F(z_0,t=0)$ since $z_0$ does not exist.
But $F(z,t=0)$, and therefore $F(v_0)$, is by definition given by
\begin{align}
F(v_0) = F(z,t=0) = \sum_m z^m p(m, t=0),
\end{align}
where $p(m, t=0)$ is whatever initial probability distribution
we specify for the problem. Finally then
\begin{align}
F(z, t) &=  \left(\frac{1-(1+(z-1)e^{-t})\xi}{1-z\xi}\right)^{\lambda}
                \left(\sum_m z^m p(m, t=0)\right)
\\
F(z, t) &=  \left(\frac{1-(1-\theta)(1+(z-1)e^{-t})}
                        {1-(1-\theta)z}\right)^{\lambda}
                \left(\sum_m z^m p(m, t=0)\right).
\end{align}
If the initial condition is $p(0,0)=1$, then the second term in
parentheses disappears and this clearly reduces to the usual negative
binomial generating function as $t\rightarrow\infty$. For any other
initial condition, this has to remain true but it's not obvious to me
how the math will work out, even for the next-simplest initial condition
like $p(m,0) = \delta_{mk}$.
\mmnote{Still need to work out some algebra details here to see what happens.}
\mmnote{Our t-dependence is already different than theirs, even though
the steady state is the same. For trivial $k=0$ IC, everything will work
out similarly, I think. For anything else, I think the direct approach
of differentiating will produce an impossible mess. But we might be able
to use their idea of a ``propagator'' probability and do a convolution
over that; in principle, getting to the same place, but that might be
tidier??}

\section{Adding translation}
If we include transcription and translation, the master equation becomes
very similar to Eq.~(1) in~\cite{Shahrezaei2008}, except with mRNA
production terms borrowed from our~\eref{eq:1state_unreg_003}.
As in~\cite{Shahrezaei2008}, it is more convenient to nondimensionalize
time by the protein degradation rate $\gamma_p$ rather than the mRNA
degradation rate $\gamma_m$.
Then $\gamma\equiv\gamma_m/\gamma_p$ is the dimensionless mRNA lifetime,
and $\lambda\equiv r_m/\gamma_p$ is the dimensionless rate at which
transcription bursts initiate.
The mean burst size $\beta = (1-\theta)/\theta$ as before.
The new parameter we need is the translation rate $r_p$,
which we immediately nondimensionalize as $r\equiv r_m/\gamma_p$.
The authors of~\cite{Shahrezaei2008} have three dimensionless parameters
$a$, $b$, and $\gamma$, which have a simple correspondence with ours:
$a=\lambda$, $b=r/\gamma$, and $\gamma=\gamma$.

With all this, the master equation takes the form
\begin{align}
\deriv{t}p(m,n,t) = 
        & \lambda \left[\sum_{m^\prime=0}^m G_{m-m^\prime}(\theta) p(m^\prime,n,t)
        - p(m,n,t) \right]
        \\
        & + r m [ p(m,n-1,t) - p(m,n,t) ]
        \\
        & + \gamma [ (m+1) p(m+1,n,t) - m p(m,n,t) ]
        \\
        & + (n+1) p(m,n+1,t) - n p(m,n,t).
\end{align}
The first line covers transcription, the second translation, the third
mRNA degradation, and the fourth protein degradation. Conceptually, this
is identical to Eq.~(1) in~\cite{Shahrezaei2008} except for the very
first term.\footnote{The equations also differ by nondimensionalization
and relabeling of parameters, but these are ``bookkeeping'' operations,
not physical/conceptual ones.}
The probability generating function is now defined as
\begin{align}
F(z,w,t) = \sum_{m,n=0}^\infty z^m w^n p(m,n,t),
\end{align}
and similarly to before, we transform the master equation by multiplying
both sides by $z^m w^n$ and summing over $m$ and $n$.
The first term is unaffected by summing over $n$ and proceeds exactly as
in the mRNA-only case. All other terms proceed exactly as
in~\cite{Shahrezaei2008}, so the PDE for the generating function is
\begin{align}
\pderiv[F]{t} = \left(\frac{\theta}{1-z(1-\theta)}-1\right)\lambda F
        + rz(w - 1) \pderiv[F]{z} + \gamma(1 - z) \pderiv[F]{z}
        + (1 - w) \pderiv[F]{w}.
\end{align}
For convenience, and analogously to the mRNA-only case, changing variables to
$u=z-1$, $v=w-1$, and $\xi=1-\theta$ and simplifying gives
\begin{align}
\pderiv[F]{t} = \frac{\lambda\xi u}{1-\xi(u+1)} F
        + \left[ rv(1+u) - \gamma u \right] \pderiv[F]{u}
        - v \pderiv[F]{v},
\end{align}
or, rearranging terms to emphasize the analogy with
Eq.~(2) in~\cite{Shahrezaei2008},
\begin{align}
\pderiv[F]{v} - \left[ r(1+u) - \gamma \frac{u}{v} \right] \pderiv[F]{u} 
        + \frac{1}{v}\pderiv[F]{t} = \frac{\lambda\xi}{1-\xi(u+1)} \frac{u}{v} F.
\end{align}
Again, the left-hand side is functionally identical to Eq.~(2)
in~\cite{Shahrezaei2008}, and only the right-hand side has new content.
Despite the more complicated right-hand side, this is still solvable
with the method of characteristics; parametrizing the characteristics by
$s$ leads to a set of four equations, namely
\begin{align}
\deriv[v]{s}=1
\\
\deriv[t]{s}=\frac{1}{v}
\\
\deriv[u]{s} = \gamma \frac{u}{v} - r(1+u)
\\
\deriv[F]{s} = \frac{\lambda\xi}{1-\xi(u+1)} \frac{u}{v} F.
\end{align}
The equations for $t$ and $v$ are identical to the mRNA-only case,
allowing us to immediately eliminate $s$ from the problem and replace it
with $v$. We also have
\begin{align}
v = v_0 e^t
\end{align}
as before, for some initial $v_0$ yet to be found corresponding to $t=0$.
The remaining two ODEs to be solved are
\begin{align}
\deriv[u]{v} = \gamma \frac{u}{v} - r(1+u)
\\
\deriv[F]{v} = \frac{\lambda\xi}{1-\xi(u+1)} \frac{u}{v} F.
\end{align}
Shahrezaei and Swain approximately solve the first for $\gamma \gg 1$.
\mmnote{Now I see why they defined $b$ as they did, so they could get
$\gamma$ alone on one side. That's a bit cleaner than the way I've
parametrized, but it shouldn't change the conclusion.} In fact, this
approximation should serve us even better than it did them: their yeast
genes of interest feature $\gamma\sim 2-6$, whereas for \textit{E. coli}
with protein ``degradation'' dominated by dilution we expect $\gamma\sim
5-15$. They find $u(v) \approx bv/(1-bv)$, which, translated to our
variables, reads
\begin{align}
u(v) \approx \frac{rv}{\gamma-rv}.
\end{align}
Plugging this into the ODE for $F$ gives
\begin{align}
\deriv[F]{v} \approx
        \frac{\lambda\xi}{1-\xi\frac{\gamma}{\gamma-rv}} \frac{r}{\gamma-rv} F
        = \frac{\lambda r \xi}{(1-\xi)\gamma - rv} F
        = \frac{\lambda r (1-\theta)}{\theta\gamma - rv} F,
\end{align}
where in the last step we changed variables back to $\theta=1-\xi$.
This is still solvable with separation of variables, as
\begin{align}
\int\frac{dF}{F} &\approx \int \frac{\lambda r (1-\theta) dv}{\theta\gamma - rv} F
\\
\ln \frac{F(v)}{F(v_0)} &\approx -\lambda (1-\theta)
                \ln\left(\frac{\theta\gamma - rv}{\theta\gamma - rv_0}\right),
\end{align}
where we handled initial conditions in close analogy to the mRNA-only case.
With some algebra, and recalling that $v=w-1$ and $v_0=(w-1)e^{-t}$, we have
\begin{align}
F(w,t) \approx \left(\frac{1-\frac{r}{\theta\gamma}(w-1)e^{-t}}
                {1-\frac{r}{\theta\gamma}(w-1)}\right)^{\lambda(1-\theta)}
                \left(\sum_n w^n p(n,t=0)\right).
\label{eq:1state_unreg_030}
\end{align}
Note that $z$, which corresponded to mRNA copy number, dropped out when
we substituted our approximate solution for $u$ as a function of $v$.
This gave us a generating function for the protein distribution alone,
with all stochasticity and burstiness of the mRNA (approximately)
incorporated. The second term is simply the initial condition set by our
initially specified probability distribution on protein.
The first term is exactly, up to relabeling parameters, Eq.~(7)
in~\cite{Shahrezaei2008}, which leads to a probability distribution that
resembles a negative binomial, but with time dependence, multiplied by a
hypergeometric funtion.

Is this result of use to us? As with Eq.~(8) in~\cite{Shahrezaei2008},
it is valid only for $\gamma \gg 1$ and $t\gg \gamma^{-1}$.
The former is easily satisfied for our system. The latter means the result
is only valid for times longer than the mRNA lifetime,
which is also satisfied in our problem.

But the time scale to approach steady state is set by the protein
lifetime itself, which, if dominated by dilution, means $F(w,t)$ will
never reach steady-state, not even to a crude approximation.
So~\eref{eq:1state_unreg_030} applies, but we \textit{cannot} take its
steady-state limit. For very special initial conditions, Shahrezaei and
Swain work out the full time-dependent probability distribution, Eq.~(8)
in~\cite{Shahrezaei2008}. But it is not clear if we can do this for the
binomially partitioned initial distribution that we would like to use.
\mmnote{Needs more thought.}

%%%%%%%%%%%%%%%%%%%%% APPENDICES %%%%%%%%%%%%%%%%%%%%%%%%%%%%%%%%%%%
\appendix


% \bibliographystyle{nature}
\bibliographystyle{abbrv}
\bibliography{../library}

\end{document}
