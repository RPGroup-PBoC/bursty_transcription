% !TEX root = ./busty_transcription.tex
\section{Derivations for non-bursty promoter models}
\label{sec:non_bursty}

\mrm{I personally think that the paper so far has done a great job at being very 
pedagogical. That is why I am of the opinion that before jumping into the 
derivation of moments and so on there needs to be a $p(m, t + \Delta t)$ kind of 
derivation.}

In this section we detail the calculation of mean mRNA levels, fold-changes in
expression, and Fano factors for nonequilibrium promoter models one through four
in Figure~\ref{fig1:means_cartoons}. These are the results that were quoted but
not derived in Sections~\ref{section_02_means} and \ref{sec:beyond_means} of the
main text. In each of these four models, the natural mathematization of their
cartoons is as a chemical master equation such as Eq.~\ref{eq:2state_rep_cme}
for model one, which for completeness we reproduce here as
\begin{align}
\begin{split}
\deriv{t}p_R(m,t) =& 
- \overbrace{k_R^- p_R(m,t)}^{R \rightarrow U}
+ \overbrace{k_R^+ p_U(m,t)}^{U \rightarrow R}
+ \overbrace{(m+1)\gamma p_R(m+1,t)}^{m + 1 \rightarrow m}
- \overbrace{\gamma mp_R(m,t)}^{m \rightarrow m - 1}
\\
\deriv{t}p_U(m,t) =&\; 
\overbrace{k_R^- p_R(m,t)}^{R \rightarrow U}
- \overbrace{k_R^+ p_U(m,t)}^{U \rightarrow R}
+ \overbrace{rp_U(m-1,t)}^{m-1 \rightarrow m}
- \overbrace{rp_U(m,t)}^{m \rightarrow m + 1}
\\
&+ \overbrace{(m+1)\gamma p_U(m+1,t)}^{m + 1 \rightarrow m}
- \overbrace{\gamma mp_U(m,t)}^{m \rightarrow m - 1}.
\label{eq:poisson_promoter_cme_appdx}
\end{split}
\end{align}
Here $p_R(m,t)$ and $p_U(m,t)$ are the probabilities of finding the system with
$m$ mRNA molecules at time $t$ either in the repressor bound or unbound states,
respectively. $r$ is the probability per unit time that a transcript will be
initiated when the repressor is unbound, and $\gamma$ is the probability per
unit time for a given mRNA to be degraded. $k_R^-$ is the probability per unit
time that a bound repressor will unbind, while $k_R^+$ is the probability per
unit time that an unbound operator will become bound by a repressor. Assuming
mass action kinetics, $k_R^+$ is proportional to repressor copy number $R$.

Next consider the cartoon for nonequilibrium model two in
Figure~\ref{fig1:means_cartoons}(C). Now we must track probabilities $p_R$,
$p_P$, and $p_E$ for the repressor bound, empty, and polymerase bound states,
respectively. By analogy to~\ref{eq:poisson_promoter_cme_appdx}, the master
equation for model two can be written
\begin{align}
\begin{split}
\deriv{t}p_R(m,t) =& 
- \overbrace{k_R^- p_R(m,t)}^{R \rightarrow U}
+ \overbrace{k_R^+ p_E(m,t)}^{U \rightarrow R}
+ \overbrace{(m+1)\gamma p_R(m+1,t)}^{m + 1 \rightarrow m}
- \overbrace{\gamma mp_R(m,t)}^{m \rightarrow m - 1}
\\
\deriv{t}p_E(m,t) =&\; 
  \overbrace{k_R^- p_R(m,t)}^{R \rightarrow U}
- \overbrace{k_R^+ p_E(m,t)}^{U \rightarrow R}
+ \overbrace{(m+1)\gamma p_E(m+1,t)}^{m + 1 \rightarrow m}
- \overbrace{\gamma mp_E(m,t)}^{m \rightarrow m - 1}.
\\
&
+ \overbrace{k_P^- p_P(m,t)}^{A \rightarrow U}
- \overbrace{k_P^+ p_E(m,t)}^{U \rightarrow A}
+ \overbrace{rp_P(m-1,t)}^{m-1 \rightarrow m,\;A \rightarrow U}
\\
\deriv{t}p_P(m,t) =&\; 
- \overbrace{k_P^- p_P(m,t)}^{A \rightarrow U}
+ \overbrace{k_P^+ p_E(m,t)}^{U \rightarrow A}
+ \overbrace{(m+1)\gamma p_P(m+1,t)}^{m + 1 \rightarrow m}
- \overbrace{\gamma mp_P(m,t)}^{m \rightarrow m - 1}.
\\
&- \overbrace{rp_P(m,t)}^{m \rightarrow m + 1,\;A \rightarrow U}.
\label{eq:model2_cme_appdx}
\end{split}
\end{align}
$k_P^+$ and $k_P^-$ are defined in close analogy to $k_R^+$ and $k_R^-$, except
for RNAP binding and unbinding instead of repressor. Similarly $p_P(m,t)$ is
defined for the active (RNAP-bound) state exactly as are $p_R(m,t)$ and
$p_E(m,t)$ for the repressor bound and unbound states, respectively. The new
subtlety Eq.~\ref{eq:model2_cme_appdx} introduces compared to
Eq.~\ref{eq:poisson_promoter_cme_appdx} is that when mRNAs are produced, the
promoter state also changes. This is encoded by the terms involving $r$, the
last term in each of the equations for $p_E$ and $p_P$. The former accounts for
arrivals in the unbound state and the latter accounts for departures from the
RNAP-bound state.

To condense and clarify the unwieldy notation of Eq.~\ref{eq:model2_cme_appdx},
it can be rewritten in matrix form. We define the column vector $\vec{p}(m,t)$
as
\begin{equation}
\vec{p}(m,t)
= \begin{pmatrix} p_R(m,t) \\ p_E(m,t) \\ p_P(m,t) \end{pmatrix}
\end{equation}
to gather, for a given $m$, the probabilities of finding the system in the three
possible promoter states. Then all the transition rates may be condensed into
matrices which multiply this vector. The first matrix is
\begin{equation}
\mathbf{K} = \begin{pmatrix} -k_R^- & k_R^+ & 0 \\
                        k_R^- & -k_R^+ -k_P^+ & k_P^- \\
                        0 & k_P^+ & -k_P^- 
                \end{pmatrix},
% \label{eq:3state_cme_matrices_pt1}
\end{equation}
which tracks all promoter state changes in Eq.~\ref{eq:model2_cme_appdx} that
are \textit{not} accompanied by a change in the mRNA copy number. The two terms
accounting for transcription, the only transition that increases mRNA copy
number, must be handled by two separate matrices given by
\begin{equation}
\mathbf{R_A} = \begin{pmatrix}
                0 & 0 & 0 \\ 
                0 & 0 & r \\ 
                0 & 0 & 0
                \end{pmatrix},\
\mathbf{R_D} = \begin{pmatrix}
                0 & 0 & 0 \\ 
                0 & 0 & 0 \\ 
                0 & 0 & r
                \end{pmatrix}.
% \label{eq:3state_cme_matrices_pt2}
\end{equation}
$\mathbf{R_A}$ accounts for transitions \textit{arriving} in a given state while
$\mathbf{R_D}$ tracks \textit{departing} transitions. With these definitions, we
can condense Eq.~\ref{eq:model2_cme_appdx} into the single equation
\begin{equation}
\frac{d}{dt} \vec{p}(m,t) =
\left( \mathbf{K} - \mathbf{R_D} - \gamma m \mathbf{I} \right) \vec{p}(m,t)
                + \mathbf{R_A} \vec{p}(m-1,t) +
                \gamma (m+1) \mathbf{I} \vec{p}(m+1,t),
\label{eq:generic_cme_appdx}
\end{equation}
which is just Eq.~\ref{eq:3state_rep_cme} in the main text. Straightforward if
tedious algebra verifies that Eqs.~\ref{eq:model2_cme_appdx}
and~\ref{eq:generic_cme_appdx} are in fact equivalent.

Although we derived Eq.~\ref{eq:generic_cme_appdx} for the particular case of
nonequilibrium model two in Figure~\ref{fig1:means_cartoons}, in fact the
chemical master equations for all of the nonequilibrium models in
Figure~\ref{fig1:means_cartoons} except for model five can be cast in this form.
(We treat model five separately in Appendix~\ref{sec:gen_fcn_appdx}.) Model
three introduces no new subtleties beyond model two and
Eq.~\ref{eq:generic_cme_appdx} applies equally well to it, simply with different
matrices of dimension four instead of three. Models one and four are both
handled by Eq.~\ref{eq:2state_rep_cme} in the main text, which is just
Eq.~\ref{eq:generic_cme_appdx} except in the special case of $\mathbf{R_D} =
\mathbf{R_A} \equiv \mathbf{R}$, since in these two models transcription
initiation events do not change promoter state.

Recalling that our goal in this section is to derive expressions for mean mRNA
and Fano factor for nonequilibrium models one through four in
Figure~\ref{fig1:means_cartoons}, and since all four of these models are
described by Eq.~\ref{eq:generic_cme_appdx}, we can save substantial effort by
deriving general formulas for mean mRNA and Fano factor from
Eq.~\ref{eq:generic_cme_appdx} once and for all. Then for each model we can
simply plug in the appropriate matrices for $\mathbf{K}$, $\mathbf{R_D}$, and
$\mathbf{R_A}$ and carry out the remaining algebra.

For our purposes it will suffice to derive the first and second moments of the
mRNA distribution from this master equation, similar to the treatment
in~\cite{Sanchez2011}, but we refer the interested reader
to~\cite{Razo-Mejia2020} for an analogous treatment demonstrating an analytical
solution for arbitrary moments.

\subsection{General forms for mean mRNA and Fano factor}
Our task now is to derive expressions for the first two moments of the
steady-state mRNA distribution from Eq.~\ref{eq:generic_cme_appdx}. Our
treatment of this is analogous to that given in~\cite{Sanchez2011}
and~\cite{Razo-Mejia2020}. We first obtain the steady-state limit of
Eq.~\ref{eq:generic_cme_appdx} in which the time derivative vanishes, giving
\begin{equation}
0 =
\left( \mathbf{K} - \mathbf{R_D} - \gamma m \mathbf{I} \right) \vec{p}(m)
                + \mathbf{R_A} \vec{p}(m-1) +
                \gamma (m+1) \mathbf{I} \vec{p}(m+1),
\label{eq:generic_cme_ss}
\end{equation}
From this, we want to compute
\begin{equation}
\langle{m}\rangle = \sum_S \sum_{m=0}^\infty m \, p_S(m)
\end{equation}
and
\begin{equation}
\langle{m^2}\rangle = \sum_S \sum_{m=0}^\infty m^2 p_S(m)
\end{equation}
which define the average values of $m$ and $m^2$ at steady state, where the
averaging is over all possible mRNA copy numbers and promoter states $S$. For
example, for model one in Figure~\ref{fig1:means_cartoons}(C), the sum on $S$
would cover repressor bound and unbound states ($R$ and $U$ respectively), for
model two, the sum would cover repressor bound, polymerase bound, and empty
states ($R$, $P$, and $E$), and so on for the other models.

Along the way it will be convenient to define the following
\textit{conditional} moments as
\begin{equation}
\langle\vec{m}\rangle = \sum_{m=0}^\infty m \vec{p}(m),
\end{equation}
and
\begin{equation}
\langle\vec{m}^2\rangle = \sum_{m=0}^\infty m^2 \vec{p}(m).
\end{equation}
These objects are vectors of the same size as $\vect{p}(m)$, and each component
can be thought of as the expected value of the mRNA copy number, or copy number
squared, conditional on the promoter being in a certain state. For example, for
model one in Figure~\ref{fig1:means_cartoons} which has repressor bound and
unbound states labeled $R$ and $U$, $\langle\vec{m}^2\rangle$ would be
\begin{equation}
\langle\vec{m}^2\rangle
= \begin{pmatrix} \sum_{m=0}^\infty m^2 p_R(m)
                \\ \sum_{m=0}^\infty m^2 p_U(m) \end{pmatrix}.
\end{equation}
Analogously to $\langle\vec{m}\rangle$ and $\langle\vec{m}^2\rangle$,
it is convenient to define the vector
\begin{equation}
\langle\vec{m}^0\rangle = \sum_{m=0}^\infty \vec{p}(m),
\end{equation}
whose elements are simply the probabilities of finding the system in each of the
possible promoter states. To reduce notational clutter in what follows, sums
over $m$ will be assumed to run from 0 to $\infty$ unless otherwise specified,
i.e., $\sum_{m=0}^\infty$ will be denoted simply as $\sum_m$. It will also be
convenient to denote by $\vec{1}^\dagger$ a row vector of the same length as
$\vec{p}$ whose elements are all 1, such that, for instance, $\vec{1}^\dagger
\cdot \langle\vec{m}^0\rangle = 1$, $\vec{1}^\dagger \cdot \langle\vec{m}\rangle
= \langle{m}\rangle$, etc.

\subsubsection{Promoter state probabilities $\langle\vec{m}^0\rangle$}
\label{sec:m0_appdx}
To begin, we will find the promoter state probabilities
$\langle\vec{m}^0\rangle$ from Eq.~\ref{eq:generic_cme_ss} by summing over all
mRNA copy numbers $m$, producing
\begin{equation}
0 = \sum_m \left[
    \left( \mathbf{K} - \mathbf{R_D} - \gamma m \mathbf{I} \right) \vec{p}(m)
                + \mathbf{R_A} \vec{p}(m-1) +
                \gamma (m+1) \mathbf{I} \vec{p}(m+1)
\right]
\end{equation}
Using the definitions of $\langle\vec{m}^0\rangle$ and $\langle\vec{m}\rangle$,
and noting the matrices $\mathbf{K}$, $\mathbf{R_D}$, and $\mathbf{R_A}$
are all independent of $m$ and can be moved outside the sum, this simplifies to
\begin{equation}
0 = (\mathbf{K} - \mathbf{R_D}) \langle\vec{m}^0\rangle
    - \gamma \langle\vec{m}\rangle + \mathbf{R_A} \sum_m \vec{p}(m-1)
    + \gamma \sum_m (m+1)\vec{p}(m+1).
\label{eq:generic_cme_deriv_020}
\end{equation}
The last two terms can be handled by reindexing the summations, transforming
them to match the definitions of $\langle\vec{m}^0\rangle$ and
$\langle\vec{m}\rangle$. For the first, noting $\vec{p}(-1)=0$ since negative
numbers of mRNA are nonsensical, we have
\begin{equation}
\sum_{m=0}^\infty \vec{p}(m-1)
= \sum_{m=-1}^\infty \vec{p}(m)
= \sum_{m=0}^\infty \vec{p}(m) = \langle\vec{m}^0\rangle.
\end{equation}
Similarly for the second,
\begin{equation}
\sum_{m=0}^\infty (m+1)\vec{p}(m+1)
= \sum_{m=1}^\infty m\vec{p}(m)
= \sum_{m=0}^\infty m\vec{p}(m) = \langle\vec{m}\rangle,
\end{equation}
which holds since in extending the lower limit from $m=1$ to $m=0$, the extra
term we added to the sum is zero. Substituting these back in
Eq.~\ref{eq:generic_cme_deriv_020}, we have
\begin{equation}
0 = (\mathbf{K} - \mathbf{R_D}) \langle\vec{m}^0\rangle
    - \gamma \langle\vec{m}\rangle + \mathbf{R_A} \langle\vec{m}^0\rangle
    + \gamma \langle\vec{m}\rangle,
\end{equation}
or simply
\begin{equation}
0 = (\mathbf{K} - \mathbf{R_D} + \mathbf{R_A}) \langle\vec{m}^0\rangle.
\label{eq:generic_cme_vecm0}
\end{equation}
So given matrices $\mathbf{K}$, $\mathbf{R_D}$, and $\mathbf{R_A}$ describing a
promoter, finding $\langle\vec{m}^0\rangle$ simply amounts to solving this set
of linear equations, subject to the normalization constraint $\vec{1}^\dagger
\cdot \langle\vec{m}^0\rangle = 1$. It will turn out to be the case that, if the
matrix $\mathbf{K} - \mathbf{R_D} + \mathbf{R_A}$ is $n$ dimensional, it will
always have only $n-1$ linearly independent equations. Including the
normalization condition provides the $n$-th linearly independent equation,
ensuring a unique solution. So when using this equation to solve for
$\langle\vec{m}^0\rangle$, we may always drop one row of the matrix equation at
our convenience and supplement the system with the normalization condition
instead.

\subsubsection{First moments $\langle\vec{m}\rangle$ and $\langle{m}\rangle$}
By analogy to the above procedure for finding $\langle\vec{m}^0\rangle$, we may
find $\langle\vec{m}\rangle$ by first multiplying Eq.~\ref{eq:generic_cme_ss} by
$m$ and then summing over $m$. Carrying out this procedure we have
\begin{equation}
0 = \sum_m m \left[
\left( \mathbf{K} - \mathbf{R_D} - \gamma m \mathbf{I} \right) \vec{p}(m)
            + \mathbf{R_A} \vec{p}(m-1) +
            \gamma (m+1) \mathbf{I} \vec{p}(m+1)
\right],
\end{equation}
and now identifying $\langle\vec{m}\rangle$ and $\langle\vec{m}^2\rangle$ gives
\begin{equation}
0 = (\mathbf{K} - \mathbf{R_D}) \langle\vec{m}\rangle
    - \gamma \langle\vec{m}^2\rangle + \mathbf{R_A} \sum_m m\vec{p}(m-1)
    + \gamma \sum_m m(m+1)\vec{p}(m+1).
\label{eq:generic_cme_deriv_030}
\end{equation}
The summations in the last two terms can be reindexed just as we did for
$\langle\vec{m}^0\rangle$, freely adding or removing terms from the sum which
are zero. For the first,
\begin{equation}
\sum_{m=0}^\infty m\vec{p}(m-1)
= \sum_{m=-1}^\infty (m+1)\vec{p}(m)
= \sum_{m=0}^\infty (m+1)\vec{p}(m)
= \langle\vec{m}\rangle + \langle\vec{m}^0\rangle,
\end{equation}
and for the second,
\begin{equation}
\sum_{m=0}^\infty m(m+1)\vec{p}(m+1)
= \sum_{m=1}^\infty (m-1)m\vec{p}(m)
= \sum_{m=0}^\infty (m-1) m\vec{p}(m)
= \langle\vec{m}^2\rangle - \langle\vec{m}\rangle.
\end{equation}
Substituting back in Eq.~\ref{eq:generic_cme_deriv_030} then produces
\begin{equation}
0 = (\mathbf{K} - \mathbf{R_D}) \langle\vec{m}\rangle
- \gamma \langle\vec{m}^2\rangle
+ \mathbf{R_A} (\langle\vec{m}\rangle + \langle\vec{m}^0\rangle)
+ \gamma (\langle\vec{m}^2\rangle - \langle\vec{m}\rangle),
\end{equation}
or after simplifying
\begin{equation}
0 = (\mathbf{K} - \mathbf{R_D} + \mathbf{R_A} - \gamma) \langle\vec{m}\rangle
+ \mathbf{R_A} \langle\vec{m}^0\rangle.
\label{eq:generic_cme_deriv_040}
\end{equation}
So like $\langle\vec{m}^0\rangle$, $\langle\vec{m}\rangle$ is also found by
simply solving a set of linear equations after first solving for
$\langle\vec{m}^0\rangle$ from Eq.~\ref{eq:generic_cme_vecm0}.

Next we can find the mean mRNA copy number $\langle{m}\rangle$ from
$\langle\vec{m}\rangle$ according to
\begin{equation}
\langle{m}\rangle = \vec{1}^\dagger\cdot\langle\vec{m}\rangle,
\label{eq:m_from_vecm}
\end{equation}
where $\vec{1}^\dagger$ is a row vector whose elements are all 1.
Eq.~\ref{eq:m_from_vecm} holds since the $i$-th element of the column vector
$\langle\vec{m}\rangle$ is the mean mRNA value conditional on the system
occupying the $i-$th promoter state, so the dot product with $\vec{1}^\dagger$
amounts to simply summing the elements of $\langle\vec{m}\rangle$. Rather than
solving Eq.~\ref{eq:generic_cme_deriv_040} for $\langle\vec{m}\rangle$ and then
computing $\vec{1}^\dagger\cdot\langle\vec{m}\rangle$, we may take a shortcut by
multiplying Eq.~\ref{eq:generic_cme_deriv_040} by $\vec{1}^\dagger$ first. The
key observation that makes this useful is that
\begin{equation}
\vec{1}^\dagger \cdot (\mathbf{K} - \mathbf{R_D} + \mathbf{R_A}) = 0.
\end{equation}
Intuitively, this equality holds because each column of this matrix
represents transitions to and from a given promoter state. In any given column,
the diagonal encodes all departing transitions and off-diagonals encode all
arriving transitions. Conservation of probability means that each column
sums to zero, and summing columns is exactly the operation that multiplying by
$\vec{1}^\dagger$ carries out.

Proceeding then in multiplying Eq.~\ref{eq:generic_cme_deriv_040} by
$\vec{1}^\dagger$ produces
\begin{equation}
0 = -\gamma \vec{1}^\dagger\cdot\langle\vec{m}\rangle
+ \vec{1}^\dagger\cdot\mathbf{R_A}\langle\vec{m}^0\rangle,
\end{equation}
or simply
\begin{equation}
\langle{m}\rangle
= \frac{1}{\gamma} \vec{1}^\dagger\cdot\mathbf{R_A}\langle\vec{m}^0\rangle.
\label{eq:generic_mean_m_appdx}
\end{equation}
We note that the in equilibrium models of transcription such as in
Figure~\ref{fig1:means_cartoons}, it is usually \textit{assumed} that the mean
mRNA level is given by the ratio of initiation rate $r$ to degradation rate
$\gamma$ multiplied by the probability of finding the system in the RNAP-bound
state. Reassuringly, we have recovered exactly this result from the master
equation picture: the product
$\vec{1}^\dagger\cdot\mathbf{R_A}\langle\vec{m}^0\rangle$ picks out the
probability of the active promoter state from $\langle\vec{m}^0\rangle$ and
multiplies it by the initiation rate $r$.

\subsubsection{Second moment $\langle{m}^2\rangle$ and Fano factor $\nu$}
Continuing the pattern of the zeroth and first moments, we now find
$\langle\vec{m}^2\rangle$ by multiplying Eq.~\ref{eq:generic_cme_ss} by $m^2$
and then summing over $m$, which explicitly is
\begin{equation}
0 = \sum_m m^2 \left[
\left( \mathbf{K} - \mathbf{R_D} - \gamma m \mathbf{I} \right) \vec{p}(m)
            + \mathbf{R_A} \vec{p}(m-1) +
            \gamma (m+1) \mathbf{I} \vec{p}(m+1)
\right].
\end{equation}
Identifying the moments $\langle\vec{m}^2\rangle$ and $\langle\vec{m}^3\rangle$
in the first term simplifies this to
\begin{equation}
0 = (\mathbf{K} - \mathbf{R_D}) \langle\vec{m}^2\rangle
    - \gamma \langle\vec{m}^3\rangle + \mathbf{R_A} \sum_m m^2\vec{p}(m-1)
    + \gamma \sum_m m^2(m+1)\vec{p}(m+1).
\label{eq:generic_cme_deriv_050}
\end{equation}
Reindexing the sums of the last two terms proceeds just as it did for the zeroth
and first moments. Explicitly, we have
\begin{equation}
\sum_{m=0}^\infty m^2\vec{p}(m-1)
= \sum_{m=-1}^\infty (m+1)^2\vec{p}(m)
= \sum_{m=0}^\infty (m+1)^2\vec{p}(m)
= \langle\vec{m}^2\rangle + 2\langle\vec{m}\rangle + \langle\vec{m}^0\rangle,
\end{equation}
for the first sum and
\begin{equation}
\sum_{m=0}^\infty m^2(m+1)\vec{p}(m+1)
= \sum_{m=1}^\infty (m-1)^2m\vec{p}(m)
= \sum_{m=0}^\infty (m-1)^2 m\vec{p}(m)
= \langle\vec{m}^3\rangle - 2\langle\vec{m}^2\rangle + \langle\vec{m}\rangle
\end{equation}
for the second. Substituting the results of the sums back in
Eq.~\ref{eq:generic_cme_deriv_050} gives
\begin{equation}
0 = (\mathbf{K} - \mathbf{R_D}) \langle\vec{m}^2\rangle
- \gamma \langle\vec{m}^3\rangle
+ \mathbf{R_A}
    (\langle\vec{m}^2\rangle + 2\langle\vec{m}\rangle + \langle\vec{m}^0\rangle)
+ \gamma
    (\langle\vec{m}^3\rangle - 2\langle\vec{m}^2\rangle + \langle\vec{m}\rangle),
\end{equation}
and after grouping like powers of $m$ we have
\begin{equation}
0 = (\mathbf{K} - \mathbf{R_D} + \mathbf{R_A} - 2\gamma) \langle\vec{m}^2\rangle
+ (2\mathbf{R_A} + \gamma) \langle\vec{m}\rangle
+ \mathbf{R_A} \langle\vec{m}^0\rangle.
\label{eq:generic_cme_deriv_060}
\end{equation}
As we found when computing $\langle{m}\rangle$ from $\langle\vec{m}\rangle$, we
can spare ourselves some algebra by multiplying
Eq.~\ref{eq:generic_cme_deriv_060} by $\vect{1}^\dagger$, which then reduces to
\begin{equation}
0 = - 2\gamma \langle{m}^2\rangle
+ \vec{1}^\dagger\cdot(2\mathbf{R_A} + \gamma) \langle\vec{m}\rangle
+ \vec{1}^\dagger\cdot\mathbf{R_A} \langle\vec{m}^0\rangle,
\end{equation}
and noting from Eq.~\ref{eq:generic_mean_m_appdx} that
$\vec{1}^\dagger\cdot\mathbf{R_A} \langle\vec{m}^0\rangle
= \gamma\langle{m}\rangle$, we have the tidy result
\begin{equation}
\langle{m}^2\rangle
= \langle{m}\rangle + \frac{1}{\gamma}
        \vec{1}^\dagger\cdot\mathbf{R_A} \langle\vec{m}\rangle.
\end{equation}

Finally we have all the preliminary results needed to write a general expression
for the Fano factor $\nu$. The Fano factor is defined as the ratio of variance
to mean, which can be written as
\begin{equation}
\nu = \frac{\langle{m}^2\rangle - \langle{m}\rangle^2}{\langle{m}\rangle}
= \frac{
    \langle{m}\rangle + \frac{1}{\gamma}
        \vec{1}^\dagger\cdot\mathbf{R_A} \langle\vec{m}\rangle
    - \langle{m}\rangle^2
    }{\langle{m}\rangle}
\end{equation}
and simplified to
\begin{equation}
\nu = 1 - \langle{m}\rangle
+ \frac{\vec{1}^\dagger\cdot \mathbf{R_A}\langle\vec{m}\rangle}
        {\gamma \langle{m}\rangle}.
\label{eq:generic_fano_appdx}
\end{equation}
Note a subtle notational trap here: $\langle{m}\rangle = \frac{1}{\gamma}
\vec{1}^\dagger\cdot\mathbf{R_A}\langle\vec{m}^0\rangle$ rather than the by-eye
similar but wrong expression $\langle{m}\rangle \ne \frac{1}{\gamma}
\vec{1}^\dagger\cdot\mathbf{R_A}\langle\vec{m}\rangle$, so the last term in
Eq.~\ref{eq:generic_fano_appdx} is in general quite nontrivial. For a generic
promoter, Eq.~\ref{eq:generic_fano_appdx} may be greater than, less than, or
equal to one, as asserted in Section~\ref{sec:beyond_means}. We have not found
the general form Eq.~\ref{eq:generic_fano_appdx} terribly intuitive and instead
defer discussion to specific examples.

\subsubsection{Summary of general results}
For ease of reference, we collect and reprint here the key results derived in
this section that are used in the main text and subsequent subsections. Mean
mRNA copy number and Fano factor are given by Eqs.~\ref{eq:generic_mean_m_appdx}
and \ref{eq:generic_fano_appdx}, which are
\begin{equation}
\langle{m}\rangle
= \frac{1}{\gamma} \vec{1}^\dagger\cdot\mathbf{R_A}\langle\vec{m}^0\rangle
\label{eq:mean_m_appdx_ref}
\end{equation}
and
\begin{equation}
\nu = 1 - \langle{m}\rangle
+ \frac{\vec{1}^\dagger\cdot \mathbf{R_A}\langle\vec{m}\rangle}
        {\gamma \langle{m}\rangle},
\label{eq:fano_appdx_ref}
\end{equation}
respectively. To compute these two quantities, we need the expressions for
$\langle\vec{m}^0\rangle$ and $\langle\vec{m}\rangle$ given by solving
Eqs.~\ref{eq:generic_cme_vecm0} and \ref{eq:generic_cme_deriv_040},
respectively, which are
\begin{equation}
(\mathbf{K} - \mathbf{R_D} + \mathbf{R_A}) \langle\vec{m}^0\rangle = 0
\label{eq:vecm0_appdx_ref}
\end{equation}
and
\begin{equation}
(\mathbf{K} - \mathbf{R_D}
+ \mathbf{R_A} - \gamma\mathbf{I}) \langle\vec{m}\rangle
= - \mathbf{R_A} \langle\vec{m}^0\rangle.
\label{eq:vecm_appdx_ref}
\end{equation}
Some comments are in order before we consider particular models. First, note
that to obtain $\langle\vec{m}\rangle$ and $\nu$, we need not bother solving for
all components of the vectors $\langle\vec{m}^0\rangle$ and
$\langle\vec{m}\rangle$, but only the components which are multiplied by nonzero
elements of $\mathbf{R_A}$. The only component of $\langle\vec{m}^0\rangle$ that
ever survives is the transciptionally active state, and for the models we
consider here, there is only ever one such state. This will save us some amount
of algebra below.

Also note that we are computing Fano factors to verify the results of
Section~\ref{sec:beyond_means}, concerning the constitutive promoter models in
Figure~\ref{fig2:constit_cartoons} which are analogs of the simple repression
models in Figure~\ref{fig1:means_cartoons}. We can translate the matrices from
the simple repression models to the constitutive case by simply substituting all
occurrences of repressor rates by zero and removing the row and column
corresponding to the repressor bound state. The results for $\langle{m}\rangle$
computed in the repressed case can be easily translated to the constitutive
case, rather than recalculating from scratch, by taking the limit
$k_R^+\rightarrow 0$, since this amounts to sending repressor copy number to
zero.

Finally, we point out that it would be possible to compute
$\langle\vec{m}^0\rangle$ more simply using the diagram methods from King and
Altman~\cite{King1956} (also independently discovered by Hill~\cite{Hill1966}).
But to our knowledge this method cannot be applied to compute
$\langle\vec{m}\rangle$ or $\nu$, so since we would need to resort to solving
the matrix equations anyways for $\langle\vec{m}\rangle$, we choose not to
introduce the extra conceptual burden of the diagram methods simply for
computing $\langle\vec{m}^0\rangle$.

\subsection{Nonequilibrium Model One - Poisson Promoter}
\subsubsection{Mean mRNA}
For nonequilibrium model one in Figure~\ref{fig1:means_cartoons}, we have
already shown the full master equation in Eq.~\ref{eq:poisson_promoter_cme} and
Eq.~\ref{eq:poisson_promoter_cme_appdx}, but for completeness we reprint it
again as
\begin{align}
\begin{split}
\deriv{t}p_R(m,t) =& 
- \overbrace{k_R^- p_R(m,t)}^{R \rightarrow U}
+ \overbrace{k_R^+ p_U(m,t)}^{U \rightarrow R}
+ \overbrace{(m+1)\gamma p_R(m+1,t)}^{m + 1 \rightarrow m}
- \overbrace{\gamma mp_R(m,t)}^{m \rightarrow m - 1}
\\
\deriv{t}p_U(m,t) =&\; 
\overbrace{k_R^- p_R(m,t)}^{R \rightarrow U}
- \overbrace{k_R^+ p_U(m,t)}^{U \rightarrow R}
+ \overbrace{rp_U(m-1,t)}^{m-1 \rightarrow m}
- \overbrace{rp_U(m,t)}^{m \rightarrow m + 1}
\\
&+ \overbrace{(m+1)\gamma p_U(m+1,t)}^{m + 1 \rightarrow m}
- \overbrace{\gamma mp_U(m,t)}^{m \rightarrow m - 1}.
\end{split}
\end{align}
This is a direct transcription of the states and rates in
Figure~\ref{fig1:means_cartoons}. This may be converted to the matrix form of
the master equation shown in Eq.~\ref{eq:generic_cme_appdx} with matrices
\begin{equation}
\vec{p}(m) = \begin{pmatrix} p_R(m) \\ p_U(m) \end{pmatrix},\
\mathbf{K} = \begin{pmatrix} -k_R^- & k_R^+ \\ k_R^- & -k_R^+ \end{pmatrix},\
\mathbf{R} = \begin{pmatrix} 0 & 0 \\ 0 & r \end{pmatrix},\
\end{equation}
where $\mathbf{R_A}$ and $\mathbf{R_D}$ are equal, so we drop the subscript and
denote both simply by $\mathbf{R}$. Since our interest is only in steady-state
we dropped the time dependence as well.

First we need $\langle\vec{m}^0\rangle$. Label its components as $p_R$ and
$p_U$, the probabilities of finding the system in either promoter state, and
note that only $p_U$ survives multiplication by $\mathbf{R}$, since
\begin{equation}
\mathbf{R} \langle\vec{m}^0\rangle
= \begin{pmatrix} 0 & 0 \\ 0 & r \end{pmatrix}
    \begin{pmatrix} p_R \\ p_U \end{pmatrix}
= \begin{pmatrix} 0 \\ r p_U \end{pmatrix},
\label{eq:Rm0_model1_appdx}
\end{equation}
so we need not bother finding $p_R$. Then we have
\begin{equation}
(\mathbf{K} - \mathbf{R_D} + \mathbf{R_A}) \langle\vec{m}^0\rangle
= \begin{pmatrix} -k_R^- & k_R^+ \\ k_R^- & -k_R^+ \end{pmatrix}
    \begin{pmatrix} p_R \\ p_U \end{pmatrix} = 0.
\end{equation}
As mentioned earlier in Section~\ref{sec:m0_appdx}, the two rows are linearly
dependent, so taking only the first row and using normalization to set $p_R =
1-p_U$ gives
\begin{equation}
-k_R^- (1-p_U) + k_R^+ p_U = 0,
\end{equation}
which is easily solved to find
\begin{equation}
p_U = \frac{k_R^-}{k_R^- + k_R^+}.
\end{equation}
Substituting this into Eq.~\ref{eq:Rm0_model1_appdx}, and the result of that
into Eq.~\ref{eq:mean_m_appdx_ref}, we have
\begin{equation}
\langle{m}\rangle = \frac{r}{\gamma} \frac{k_R^-}{k_R^- + k_R^+}
\end{equation}
as asserted in Eq.~\ref{eq:mean_m_model1} of the main text.

\subsubsection{Fano factor}
To verify that the Fano factor for model one in
Figure~\ref{fig2:constit_cartoons}(A) is in fact 1 as claimed in the main text,
note that in this limit $p_U = 1$ and $\langle{m}\rangle = r/\gamma$. All
elements of $\mathbf{K}$ are zero, and $\mathbf{R_A}-\mathbf{R_D} = 0$, so
Eq.~\ref{eq:vecm_appdx_ref} reduces to
\begin{equation}
- \gamma \langle\vec{m}\rangle = - r,
\end{equation}
or, in other words, since there is only one promoter state,
$\langle\vec{m}\rangle = \langle{m}\rangle$. Then it follows that
\begin{equation}
\nu = 1 -\frac{r}{\gamma}
    + \frac{r \langle{m}\rangle}{\gamma \langle{m}\rangle}
= 1
\end{equation}
as claimed.

\subsection{Nonequilibrium Model Two - RNAP Bound and Unbound States}
\subsubsection{Mean mRNA}
As shown earlier, the full master equation for model two in
Figure~\ref{fig1:means_cartoons} is
\begin{align}
\begin{split}
\deriv{t}p_R(m,t) =& 
- \overbrace{k_R^- p_R(m,t)}^{R \rightarrow U}
+ \overbrace{k_R^+ p_E(m,t)}^{U \rightarrow R}
+ \overbrace{(m+1)\gamma p_R(m+1,t)}^{m + 1 \rightarrow m}
- \overbrace{\gamma mp_R(m,t)}^{m \rightarrow m - 1}
\\
\deriv{t}p_E(m,t) =&\; 
    \overbrace{k_R^- p_R(m,t)}^{R \rightarrow U}
- \overbrace{k_R^+ p_E(m,t)}^{U \rightarrow R}
+ \overbrace{(m+1)\gamma p_E(m+1,t)}^{m + 1 \rightarrow m}
- \overbrace{\gamma mp_E(m,t)}^{m \rightarrow m - 1}.
\\
&
+ \overbrace{k_P^- p_P(m,t)}^{A \rightarrow U}
- \overbrace{k_P^+ p_E(m,t)}^{U \rightarrow A}
+ \overbrace{rp_P(m-1,t)}^{m-1 \rightarrow m,\;A \rightarrow U}
\\
\deriv{t}p_P(m,t) =&\; 
- \overbrace{k_P^- p_P(m,t)}^{A \rightarrow U}
+ \overbrace{k_P^+ p_E(m,t)}^{U \rightarrow A}
+ \overbrace{(m+1)\gamma p_P(m+1,t)}^{m + 1 \rightarrow m}
- \overbrace{\gamma mp_P(m,t)}^{m \rightarrow m - 1}.
\\
&- \overbrace{rp_P(m,t)}^{m \rightarrow m + 1,\;A \rightarrow U},
\end{split}
\end{align}
which can be condensed to the matrix form of Eq.~\ref{eq:generic_cme_appdx} with
matrices given by
\begin{equation}
\mathbf{K} = \begin{pmatrix} -k_R^- & k_R^+ & 0 \\
                        k_R^- & -k_R^+ -k_P^+ & k_P^- \\
                        0 & k_P^+ & -k_P^- 
                \end{pmatrix},\
\mathbf{R_A} = \begin{pmatrix}
                0 & 0 & 0 \\ 
                0 & 0 & r \\ 
                0 & 0 & 0
                \end{pmatrix},\
\mathbf{R_D} = \begin{pmatrix}
                0 & 0 & 0 \\ 
                0 & 0 & 0 \\ 
                0 & 0 & r
                \end{pmatrix}.
\label{eq:model2_matrices_appdx}
\end{equation}
As for model one, we must first find $\mathbf{R_A} \langle\vect{m}^0\rangle$.
Denote its components as $p_R$, $p_E$, $p_P$, the probabilities of being found
in repressor bound, empty, or RNAP-bound states, respectively. Only $p_P$ is
necessary to find since
\begin{equation}
\mathbf{R_A} \langle\vec{m}^0\rangle
= \begin{pmatrix} 0 \\ r p_P \\ 0 \end{pmatrix}.
\label{eq:model2_deriv_appdx_020}
\end{equation}
Then Eq.~\ref{eq:vecm0_appdx_ref} for $\langle\vect{m}\rangle$ reads
\begin{equation}
\begin{pmatrix} -k_R^- & k_R^+ & 0 \\
        k_R^- & -k_R^+ -k_P^+ & k_P^- + r\\
        0 & k_P^+ & -k_P^- - r
\end{pmatrix}
\begin{pmatrix}
    p_R \\ p_E \\ p_P
\end{pmatrix}
= 0.
\label{eq:model2_K-R+R_for_m0}
\end{equation}
Discarding the middle row as redundant and incorporating the normalization
condition leads to a set of three linearly independent equations, namely
\begin{align}
-k_R^- p_R + k_R^+ p_E &= 0 \\
k_P^+ p_E + (-k_P^- - r) p_P &= 0 \\
p_R + p_E + p_P &= 1.
\end{align}
Using $p_R = 1 - p_E - p_P$ to eliminate $p_R$ in the first and solving the
resulting equation for $p_E$ gives
$p_E = (1 - p_P){k_R^-}/{(k_R^- + k_R^+)}$.
Substituting this for $p_E$ in the second equation gives an equation in
$p_P$ alone which is
\begin{equation}
k_P^+ k_R^- (1-p_P) - (k_P^- + r)(k_R^+ + k_R^-) p_P = 0
\end{equation}
and solving for $p_P$ gives
\begin{equation}
p_P = \frac{k_P^+ k_R^-}{k_P^+ k_R^- + (k_P^- + r)(k_R^+ + k_R^-)}.
\end{equation}
Substituting this in Eq.~\ref{eq:model2_deriv_appdx_020} and multiplying by
$\mathbf{R_A}$ produces
\begin{equation}
\mathbf{R_A} \langle\vec{m}^0\rangle
= r \frac{k_P^+ k_R^-} {k_P^+ k_R^- + (k_P^- + r)(k_R^+ + k_R^-)}
\begin{pmatrix} 0 \\ 1 \\ 0 \end{pmatrix}
\end{equation}
from which $\langle{m}\rangle$ follows readily,
\begin{equation}
\langle{m}\rangle = \frac{r}{\gamma}
        \frac{k_P^+ k_R^-} {k_P^+ k_R^- + (k_P^- + r)(k_R^+ + k_R^-)},
\label{eq:model2_meanm_appdx}
\end{equation}
as claimed in Eq.~\ref{eq:model2_meanm} in the main text.

\subsubsection{Fano factor}
To compute the Fano factor, we first remove the repressor bound state from the
matrices describing the model, which reduce to
\begin{equation}
\mathbf{K} = \begin{pmatrix}
                -k_P^+ & k_P^- \\
                 k_P^+ &-k_P^- 
                \end{pmatrix},\
\mathbf{R_A} = \begin{pmatrix}
                0 & r \\ 
                0 & 0
                \end{pmatrix},\
\mathbf{R_D} = \begin{pmatrix}
                0 & 0 \\ 
                0 & r
                \end{pmatrix}.    
\end{equation}
Similarly we remove the repressor bound state from $\mathbf{R_A}
\langle\vec{m}^0\rangle$ and take the $k_R^+\rightarrow 0$ limit, which
simplifies to 
\begin{equation}
\mathbf{R_A} \langle\vec{m}^0\rangle
= r \frac{k_P^+ } {k_P^+ + k_P^- + r}
\begin{pmatrix} 1 \\ 0 \end{pmatrix}.
\end{equation}
Then we must compute $\langle\vec{m}\rangle$ from Eq.~\ref{eq:vecm_appdx_ref},
which with these matrices reads
\begin{equation}
(\mathbf{K} - \mathbf{R_D} + \mathbf{R_A} - \gamma\mathbf{I})
    \langle\vec{m}\rangle
= \begin{pmatrix}
    -k_P^+ -\gamma & k_P^- + r\\
    k_P^+          &-k_P^- - r - \gamma
    \end{pmatrix}
    \begin{pmatrix} m_E \\ m_P \end{pmatrix}
= r \frac{k_P^+ } {k_P^+ + k_P^- + r}
    \begin{pmatrix} 1 \\ 0 \end{pmatrix},
\end{equation}
where we labeled the components of $\langle\vec{m}\rangle$ as $m_E$ and $m_P$,
since they are the mean mRNA counts conditional upon the system residing in the
empty or polymerase bound states, respectively. Unlike for
$\langle\vec{m}^0\rangle$, the rows of this matrix are linearly independent so
we simply solve this matrix equation as is. We can immediately eliminate $m_E$
since $m_E = m_P (k_P^- + r + \gamma)/k_P^+$ from the second row, and
substituting into the first row gives an equation for $m_P$ alone, which is
\begin{equation}
\left[-(k_P^+ + \gamma)(k_P^- + r + \gamma) + k_P^+(k_P^- + r)\right] m_P
= - \frac{r (k_P^+)^2}{k_P^+ + k_P^- + r}.
\end{equation}
Expanding the products cancels several terms, and solving for $m_P$ gives
\begin{equation}
m_P = \frac{r (k_P^+)^2}
            {\gamma(k_P^+ + k_P^- + r)(k_P^+ + k_P^- + r + \gamma)}.
\end{equation}
Note then that $\vec{1}^\dagger\cdot\mathbf{R_A}\langle\vec{m}\rangle = rm_P$.
We also need the constitutive limit of $\langle{m}\rangle$ from
Eq.~\ref{eq:model2_meanm_appdx}, again found by taking $k_R^+\rightarrow0$,
which is
\begin{equation}
\langle{m}\rangle = \frac{r}{\gamma} \frac{k_P^+ } {k_P^+ + k_P^- + r}
\end{equation}
and substituting this along with
$\vec{1}^\dagger\cdot\mathbf{R_A}\langle\vec{m}\rangle = rm_P$ into
Eq.~\ref{eq:fano_appdx_ref} for the Fano factor $\nu$, we find
\begin{equation}
\nu = 1 - \frac{r}{\gamma} \frac{k_P^+ } {k_P^+ + k_P^- + r}
    + \frac{r}{\gamma}\frac{r (k_P^+)^2}{\gamma(k_P^+ + k_P^- + r)
                                        (k_P^+ + k_P^- + r + \gamma)}
\left(\frac{r}{\gamma} \frac{k_P^+ } {k_P^+ + k_P^- + r}\right)^{-1}.
\end{equation}
This simplifies to
\begin{equation}
\nu = 1 - \frac{r}{\gamma}
    \left(
        \frac{k_P^+ } {k_P^+ + k_P^- + r}
        - \frac{k_P^+ } {k_P^+ + k_P^- + r + \gamma}
    \right),
\end{equation}
which further simplifies to
\begin{equation}
\nu = 1 - \frac{r k_P^+ } {(k_P^+ + k_P^- + r)(k_P^+ + k_P^- + r + \gamma)},
\end{equation}
exactly Eq.~\ref{eq:model2_fano} in the main text.
    
\subsection{Nonequilibrium Model Three - Multistep Transcription Initiation and
Escape}

\subsubsection{Mean mRNA}
In close analogy to model two above, nonequilibrium model three from
Figure~\ref{fig1:means_cartoons}(C) can be described by our generic master
equation Eq.~\ref{eq:generic_cme_appdx} with promoter transition matrix given by
\begin{equation}
\mathbf{K} =
\begin{pmatrix} -k_R^- & k_R^+ & 0 & 0\\
        k_R^- & -k_R^+ -k_P^+ & k_P^- & 0 \\
        0 & k_P^+ & -k_P^- - k_O & 0 \\
        0 & 0 & k_O & 0
\end{pmatrix}
\end{equation}
and transcription matrices given by
\begin{equation}
\mathbf{R_A} =
\begin{pmatrix}
        0 & 0 & 0 & 0 \\ 
        0 & 0 & 0 & r \\ 
        0 & 0 & 0 & 0 \\ 
        0 & 0 & 0 & 0
\end{pmatrix},\
\mathbf{R_D} =
\begin{pmatrix}
        0 & 0 & 0 & 0 \\ 
        0 & 0 & 0 & 0 \\ 
        0 & 0 & 0 & 0 \\ 
        0 & 0 & 0 & r
\end{pmatrix}.
\end{equation}
$\langle\vec{m}^0\rangle$ is again given by Eq.~\ref{eq:vecm0_appdx_ref},
which in this case takes the form
\begin{equation}
(\mathbf{K} - \mathbf{R_D} + \mathbf{R_A}) \langle\vec{m}^0\rangle =
\begin{pmatrix} -k_R^- & k_R^+ & 0 & 0\\
    k_R^- & -k_R^+ -k_P^+ & k_P^- & r \\
    0 & k_P^+ & -k_P^- - k_O & 0 \\
    0 & 0 & k_O & - r
\end{pmatrix}
\begin{pmatrix} p_R \\ p_E \\ p_C \\ p_O
\end{pmatrix} = 0,
\end{equation}
where the four components of $\langle\vec{m}^0\rangle$ correspond to the four
promoter states repressor bound, empty, RNAP-bound closed complex, and
RNAP-bound open complex. As explained in Section~\ref{sec:m0_appdx}, we are free
to discard one linearly dependent row from this matrix and replace it with the
normalization condition $p_R + p_E + p_C + p_O = 1$. Using normalization to
eliminate $p_R$ from the first row gives
\begin{equation}
p_E = (1 - p_C - p_O)\frac{k_R^-}{k_R^- + k_R^+}.
\end{equation}
If we substitute this in the third row, then the last two rows constitute two
equations in $p_C$ and $p_O$ given by
\begin{align}
k_P^+k_R^-(1-p_C-p_O) - (k_P^- + k_O)(k_R^+ + k_R^-) p_C &= 0
\\
k_O p_C - r p_O &= 0.
\end{align}
Solving for $p_C = p_O r/k_O$ in the second and substituting into the first
gives us our desired single equation in the single variable $p_O$, which is
\begin{equation}
k_P^+k_R^- - k_P^+k_R^-\left(1 + \frac{r}{k_O}\right)p_O
            - (k_P^- + k_O)(k_R^+ + k_R^-) \frac{r}{k_O}p_O = 0,
\end{equation}
and solving for $p_O$ we find
\begin{equation}
p_O = \frac{k_P^+ k_R^- k_O}{k_P^+ k_R^- k_O + r k_P^+ k_R^- +
                            r (k_P^- + k_O) (k_R^+ + k_R^-)}.
\label{eq:model3_pO}
\end{equation}
Once again $p_O$, the transcriptionally active state, is the only component of
$\langle\vec{m}^0\rangle$ that survives multiplication by $\mathbf{R_A}$, and
$\mathbf{R_A}\langle\vec{m}^0\rangle = r p_O$. So
\begin{equation}
\langle{m}\rangle =
    \frac{1}{\gamma}\vec{1}^\dagger\cdot\mathbf{R_A}\langle\vec{m}^0\rangle
= \frac{r}{\gamma}
    \frac{k_P^+ k_R^- k_O}{k_P^+ k_R^- k_O + r k_P^+ k_R^- +
                            r (k_P^- + k_O) (k_R^+ + k_R^-)},
\end{equation}
which equals Eq.~\ref{eq:model3_mean_m} in the main text.

\subsubsection{Fano factor}
To compute the Fano factor of the analogous constitutive promoter, we first
excise the repressor states and rates from the problem. More precisely, we
construct the matrix $(\mathbf{K} - \mathbf{R_D} + \mathbf{R_A} -
\gamma\mathbf{I})$ and substitute it into Eq.~\ref{eq:vecm_appdx_ref} which is
now
\begin{equation}
(\mathbf{K} - \mathbf{R_D} + \mathbf{R_A} - \gamma\mathbf{I})
    \langle\vec{m}\rangle
= \begin{pmatrix}
    -k_P^+ - \gamma & k_P^- & r \\
     k_P^+ & -k_P^- - k_O - \gamma & 0 \\
     0 & k_O & - r- \gamma
\end{pmatrix}
\begin{pmatrix} m_E \\ m_C \\ m_O \end{pmatrix}
= -r p_O \begin{pmatrix}1 \\ 0 \\ 0 \end{pmatrix}
\end{equation}
where we labeled the unbound, closed complex, and open complex components of
$\langle\vec{m}\rangle$ as $m_E$, $m_C$, and $m_O$, respectively. $p_O$ is given
by the limit of Eq.~\ref{eq:model3_pO} as $k_R^+\rightarrow 0$, which is
\begin{equation}
p_O = \frac{k_P^+  k_O}{k_P^+ (k_O + r) + r (k_P^- + k_O)}
\equiv \frac{k_P^+  k_O}{\mathcal{Z}},
\end{equation}
where we define $\mathcal{Z}$ for upcoming convenience as this sum of terms will
appear repeatedly. We can use the second equation to eliminate $m_E$, finding
$m_E = m_C(k_P^- + k_O + \gamma)/k_P^+$, and the third to eliminate $m_C$, which
is simply $m_C = m_O(r+\gamma)/k_O$. Substituting these both into the first
equation gives a single equation for the variable of interest, $m_O$,
\begin{equation}
-(k_P^+ + \gamma) (k_P^- + k_O + \gamma) (r + \gamma) m_O
    + k_P^- k_P^+ (r + \gamma) m_O + r k_P^+ k_O m_O = - r k_P^+ k_O p_O,
\end{equation}
which is solved for $m_O$ to give
\begin{equation}
m_O = p_O \frac{r k_P^+ k_O}
    {(k_P^+ + \gamma) (k_P^- + k_O + \gamma) (r + \gamma)
        -r k_P^+ k_O - k_P^- k_P^+ (r + \gamma)}.
\end{equation}
Expanding the denominator and canceling terms leads to
\begin{equation}
m_O = p_O \frac{r}{\gamma} \frac{k_P^+ k_O}
    {\mathcal{Z} + \gamma(k_P^+ + k_P^- + k_O + r) + \gamma^2}.
\end{equation}
Now $\vec{1}^\dagger\cdot\mathbf{R_A}\langle\vec{m}\rangle = r m_O$, and
$\langle{m}\rangle = rp_O/\gamma$, so if we substitute these two quantities into
Eq.~\ref{eq:fano_appdx_ref}, we will readily obtain the Fano factor as
\begin{equation}
\nu = 1 - \langle{m}\rangle
    + \frac{\vec{1}^\dagger\cdot\mathbf{R_A}\langle\vec{m}\rangle}
            {\gamma \langle{m}\rangle}
= 1 - \frac{r}{\gamma}p_O + \frac{m_O}{p_O}.
\end{equation}
Substituting, we see that
\begin{equation}
\nu = 1 - \frac{r}{\gamma} \frac{k_P^+ k_O}{\mathcal{Z}}
    + \frac{r}{\gamma}
    \frac{k_P^+ k_O}
            {\mathcal{Z} + \gamma(k_P^+ + k_P^- + k_O + r) + \gamma^2},
\end{equation}
and after simplifying, we obtain
\begin{equation}
\nu = 1 - \frac{r k_P^+ k_O}{\mathcal{Z}}
        \frac{k_P^+ + k_P^- + k_O + r + \gamma}
            {\mathcal{Z} + \gamma(k_P^+ + k_P^- + k_O + r) + \gamma^2},
\end{equation}
as stated in Eq.~\ref{eq:model3_fano} in the main text.
        
\subsection{Nonequilibrium Model Four - ``Active'' and ``Inactive'' States}
\subsubsection{Mean mRNA}
The mathematical specification of this model is almost identical to model two.
The matrix $\mathbf{K}$ is identical, as is $\mathbf{R_D}$. The only difference
is that now $\mathbf{R_A}=\mathbf{R_D}$, i.e., both are diagonal, in contrast to
model two where $\mathbf{R_A}$ has an off-diagonal element, as in
Eq.~\ref{eq:model2_matrices_appdx}. Then the analog of
Eq.~\ref{eq:model2_K-R+R_for_m0} for finding $\langle{m}^0\rangle$ is
\begin{equation}
\begin{pmatrix} -k_R^- & k_R^+ & 0 \\
        k_R^- & -k_R^+ -k_P^+ & k_P^-\\
        0 & k_P^+ & -k_P^-
\end{pmatrix}
\begin{pmatrix}
    p_R \\ p_E \\ p_P
\end{pmatrix}
= 0.
\end{equation}
In fact we need not do this calculation explicitly and can instead recycle the
calculation of mean mRNA $\langle{m}\rangle$ from model two. The matrices are
identical except for the relabeling $k_P^- \longleftrightarrow (k_P^- + r)$, and
a careful look through the derivation of $\langle{m}\rangle$ for model two shows
that the parameters $k_P^-$ and $r$ only ever appear as the sum $k_P^- + r$. So
taking $\langle{m}\rangle$ from model two, Eq.~\ref{eq:model2_meanm_appdx}, and
relabeling $(k_P^- + r) \rightarrow k_P^-$ gives us our answer for model four,
simply
\begin{equation}
\langle{m}\rangle = \frac{r}{\gamma}
        \frac{k_P^+ k_R^-} {k_P^+ k_R^- + k_P^- (k_R^+ + k_R^-)}.
\end{equation}

\subsubsection{Fano factor}
Likewise, for computing the Fano factor of this model we may take a shortcut.
Consider the constitutive model four from Figure~\ref{fig2:constit_cartoons} for
which we want to compute the Fano factor and compare it to nonequilibrium model
one of simple repression in Figure~\ref{fig1:means_cartoons}. Mathematically
these are exactly the same model, just with rates labeled differently and the
meaning of the promoter states interpreted differently. Furthermore,
nonequilibrium model one from Figure~\ref{fig1:means_cartoons} was the model
considered by Jones et.\ al.~\cite{Jones2014}, where they derived the Fano
factor for that model to be
\begin{equation}
\nu = 1 + \frac{r k_R^+}{(k_R^+ + k_R^-)(k_R^+ + k_R^- + \gamma)}.
\end{equation}
So recognizing that the relabelings $k_R^+ \rightarrow k^-$ and
$k_R^- \rightarrow k^+$ will translate this result to our model four from
Figure~\ref{fig2:constit_cartoons}, we can immediately write down our Fano
factor as
\begin{equation}
\nu = 1 + \frac{r k^-}{(k^- + k^+)(k^- + k^+ + \gamma)},
\end{equation}
as quoted in Eq.~\ref{eq:model4_fano} and in Figure~\ref{fig2:constit_cartoons}.