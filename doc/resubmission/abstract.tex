% !TEX root = ./busty_transcription.tex
\begin{abstract}
The study of transcription remains one of the centerpieces of modern biology
with implications in settings from development to metabolism to evolution to
disease. Precision measurements using a host of different techniques including
fluorescence and sequencing readouts have raised the bar for what it means to
quantitatively understand transcriptional regulation. In particular our
understanding of the simplest genetic circuit is sufficiently refined both
experimentally and theoretically that it has become possible to carefully
discriminate between different conceptual pictures of how this regulatory system
works. This regulatory motif, originally posited by Jacob and Monod in the
1960s, consists of a single transcriptional repressor binding to a promoter site
and inhibiting transcription. In this paper, we show how seven distinct models
of this so-called simple-repression motif, based both on thermodynamic and
kinetic thinking, can be used to derive the predicted levels of gene expression
and shed light on the often surprising past success of the thermodynamic models.
These different models are then invoked to confront a variety of different data
on mean, variance and full gene expression distributions, illustrating the
extent to which such models can and cannot be distinguished, and suggesting a
two-state model with a distribution of burst sizes as the most potent of the
seven for describing the simple-repression motif.
\end{abstract}