% !TEX root = ./busty_transcription.tex
\section{Mean Gene Expression}\label{section_02_means}

As noted in the previous section, there are two broad classes of models in play
for computing the input-output functions of regulatory architectures as shown in
Figure~\ref{fig1:means_cartoons}. In both classes of model, the promoter is
imagined to exist in a discrete set of states of occupancy, with each such state
of occupancy accorded its own rate of transcription --including no
transcription for many of these states. This discretization of a potentially
continuous number of promoter states (due to effects such as supercoiling of
DNA~\cite{Chong2014, Sevier2016} or DNA looping \cite{Boedicker2013a}) is
analogous to how the Monod-Wyman-Changeux model of allostery coarse-grains
continuous molecule conformations into a finite number of
states~\cite{Martins2011}. The models are probabilistic with each state assigned
some probability and the overall rate of transcription given by 
\begin{equation}
\mbox{average rate of transcription} = \sum_i r_i p_i,
\label{eq:transcrip_prop_pbound}
\end{equation}
where $i$ labels the distinct states, $p_i$ is the probability of the
$i^{\text{th}}$ state, and $r_i$ is the rate of transcription of that state.
Ultimately, the different models differ along several key aspects: what states
to consider and how to compute the probabilities of those states.

The first class of models that are the subject of the present section focus on
predicting the mean level of gene expression. These models, sometimes known as
thermodynamic models, invoke the tools of equilibrium statistical mechanics to
compute the probabilities of the promoter microstates~\cite{Ackers1982,
Shea1985, Buchler2003, Vilar2003a, Vilar2003b, Bintu2005a, Bintu2005c,
Gertz2009, Sherman2012, Saiz2013}. As seen in
Figure~\ref{fig1:means_cartoons}(B), even within the class of thermodynamic
models, we can make different commitments about the underlying microscopic
states of the promoter. Model 1 considers only two states: a state in which a
repressor (with copy number $R$) binds to an operator and a transcriptionally
active state. The free energy difference between the repressor binding the
operator, i.e. a specific binding site, and one of the $N_{NS}$ non-specific
sites is given by $\Delta\varepsilon_R$ (given in $k_BT$ units with $\beta\equiv
(k_BT)^{-1}$). Model 2 expands this model to include an empty promoter where no
transcription occurs, as well as a state in which one of the $P$ RNAPs binds to
the promoter with binding energy $\Delta\varepsilon_P$. Indeed, the list of
options considered here does not at all exhaust the suite of different
microscopic states we can assign to the promoter. The essence of thermodynamic 
models is to assign a discrete set of states and to use equilibrium statistical 
mechanics to compute the probabilities of occupancy of those states.

The second class of models that allow us to access the mean gene expression use
chemical master equations to compute the probabilities of the different
microscopic states ~\cite{Ko1991, Peccoud1995, Record1996, Kepler2001,
Sanchez2008, Shahrezaei2008, Sanchez2011, Michel2010}. The main differences
between both modeling approaches can be summarized as: 1) Although for both
classes of models the steps involving transcriptional events are assumed to be
strictly irreversible, thermodynamic models force the regulation, i.e., the
control over the expression exerted by the repressor, to be in equilibrium. This
does not need to be the case for kinetic models. 2) Thermodynamic models ignore
the mRNA count from the state of the Markov process, while kinetic models keep
track of both the promoter state and the mRNA count. 3) Finally, thermodynamic
and kinetic models coarse-grain to different degrees the molecular mechanisms
through which RNAP enters the transcriptional event. As seen in
Figure~\ref{fig1:means_cartoons}(C), we consider a host of different kinetic
models, each of which will have its own result for both the mean (this section)
and noise (next section) in gene expression.

\subsection{Fold-changes are indistinguishable across models}
As a first stop on our search for the ``right'' model of simple repression, let
us consider what we can learn from theory and experimental measurements on the
average level of gene expression in a population of cells. One experimental
strategy that has been particularly useful (if incomplete since it misses out on
gene expression dynamics) is to measure the fold-change in mean
expression~\cite{Garcia2011}. The fold-change $FC$ is defined as
\begin{equation}
FC(R)
= \frac{\langle \text{gene expression with }R > 0 \rangle}
        {\langle \text{gene expression with }R = 0 \rangle}
= \frac{\langle m (R) \rangle}{\langle m (0) \rangle}
= \frac{\langle p (R) \rangle}{\langle p (0) \rangle},
\label{eq:fc_def}
\end{equation}
where angle brackets $\left\langle \cdot \right\rangle$ denote the average over
a population of cells and mean mRNA $\langle m\rangle$ and mean protein $\langle
p\rangle$ are viewed as a function of repressor copy number $R$. What this means
is that the fold-change in gene expression is a relative measurement of the
effect of the transcriptional repressor ($R > 0$) on the gene expression level
compared to an unregulated promoter ($R = 0$). The third equality in
Eq.~\ref{eq:fc_def} follows from assuming that the translation efficiency, i.e.,
the number of proteins translated per mRNA, is the same in both conditions. In
other words, we assume that mean protein level is proportional to mean mRNA
level, and that the proportionality constant is the same in both conditions and
therefore cancels out in the ratio. This is reasonable since the cells in the
two conditions are identical except for the presence of the transcription
factor, and the model assumes that the transcription factor has no direct effect
on translation.

Fold-change has proven a very convenient observable in past
work~\cite{Garcia2011a, Brewster2014, Razo-Mejia2018, Chure2019}. Part of its
utility in dissecting transcriptional regulation is its ratiometric nature,
which removes many secondary effects that are present when making an absolute
gene expression measurement. Also, by measuring otherwise identical cells with
and without a transcription factor present, any biological noise common to both
conditions can be made to cancel out. Figure~\ref{fig1:means_cartoons}(B) and
(C) depicts a smorgasbord of mathematicized cartoons for simple repression using
both thermodynamic and kinetic models, respectively, that have appeared in
previous literature. For each cartoon, we calculate the fold-change in mean gene
expression as predicted by that model, deferring most algebraic details to
Appendix~\ref{sec:non_bursty}. What we will find is that for all cartoons the
fold-change can be written as a Fermi function of the form
\begin{equation}
FC(R) = \left( 1 + \exp(-\Delta F_R(R) + \log(\rho))  \right)^{-1},
 \label{eq:deltaFR_eq_noneq_equiv}
\end{equation}
where the effective free energy contains two terms: the parameters $\Delta F_R$,
an effective free energy parametrizing the repressor-DNA interaction, and
$\rho$, a term derived from the level of coarse-graining used to model all
repressor-free states. In other words, the effective free energy of the Fermi
function can be written as the additive effect of the regulation given by the
repressor via $\Delta F_R$, and the kinetic scheme used to describe the steps
that lead to a transcriptional event via $\log(\rho)$ (See
Figure~~\ref{fig1:means_cartoons}(D), left panel). This implies all models
collapse to a single master curve as shown in
Figure~\ref{fig1:means_cartoons}(D). We will offer some intuition for why this
master curve exists and discuss why at the level of the mean expression, we are
unable to discriminate ``right'' from ``wrong'' cartoons given only measurements
of fold-changes in expression.

\subsubsection{Two- and Three-state Thermodynamic Models}
We begin our analysis with models 1 and 2 in
Figure~\ref{fig1:means_cartoons}(B). In each of these models the promoter
is idealized as existing in a set of discrete states; the difference being
whether or not the RNAP bound state is included or not. Gene expression is then
assumed to be proportional to the probability of the promoter being in either
the empty state (model 1) or the RNAP-bound state (model (2)). We direct the
reader to Appendix~\ref{sec:non_bursty} for details on the derivation of the
fold-change. For our purposes here, it suffices to state that the functional
form of the fold-change for model 1 is
\begin{equation}
FC(R)
= \left(1 + \frac{R}{N_{NS}} e^{-\beta\Delta\varepsilon_R}\right)^{-1},
\end{equation}
where $R$ is the number of repressors per cell, $N_{NS}$ is the number of
non-specific binding sites where the repressor can bind, $\Delta\varepsilon_R$
is the repressor-operator binding energy, and $\beta \equiv (k_BT)^{-1}$. This
equation matches the form of the master curve in
Figure~\ref{fig1:means_cartoons}(D) with $\rho=1$ and $\Delta F_R =
\beta\Delta\varepsilon_r - \log (R / N_{NS})$. For model 2 we have a similar
situation. The fold-change takes the form
\begin{eqnarray}
FC(R)
&=& \left(
1 + \frac{\frac{R}{N_{NS}} e^{-\beta\Delta\varepsilon_R}}
        {1 + \frac{P}{N_{NS}} e^{-\beta\Delta\varepsilon_P}}
\right)^{-1}
\\
&=& (1 + \exp(-\Delta F_R + \log\rho))^{-1},
\end{eqnarray}
where $P$ is the number of RNAP per cell, and $\Delta\varepsilon_P$ is the
RNAP-promoter binding energy. For this model we have $\Delta F_R =
\beta\Delta\varepsilon_R - \log(R/N_{NS})$ and $\rho = 1 +
\frac{P}{N_{NS}}\mathrm{e}^{-\beta\Delta\varepsilon_P}$. Thus far, we see that
the two thermodynamic models, despite making different coarse-graining
commitments, result in the same functional form for the fold-change in mean gene
expression.  We now explore how kinetic models fare when faced with computing
the same observable.

\subsubsection{Kinetic models}
One of the main difference between models shown in
Figure~\ref{fig1:means_cartoons}(C), cast in the language of chemical master
equations, compared with the thermodynamic models discussed in the previous
section is the probability space over which they are built. Rather than keeping
track only of the microstate of the promoter, and assuming that gene expression
is proportional to the probability of the promoter being in a certain
microstate, chemical master equation models are built on the entire probability
state of both the promoter microstate, and the current mRNA count. Therefore, in
order to compute the fold-change, we must compute the mean mRNA count on each of
the promoter microstates, and add them all together~\cite{Sanchez2013}.

Again, we consign all details of the derivation to
Appendix~\ref{sec:non_bursty}. Here we just highlight the general findings for
all five kinetic models. As already shown in Figure~\ref{fig1:means_cartoons}(C)
and (D), all the kinetic models explored can be collapsed onto the master curve.
Given that the repressor-bound state only connects to the rest of the promoter
dynamics via its binding and unbinding rates, $k_R^+$ and $k_R^-$ respectively,
all models can effectively be separated into two categories: a single
repressor-bound state, and all other promoter states with different levels of
coarse graining. This structure then guarantees that, at steady-state, detailed
balance between these two groups is satisfied. What this implies is that the
steady-state distribution of each of the non-repressor states has the same
functional form with or without the repressor, allowing us to write the
fold-change as a product of the ratio of the binding and unbinding rates of the
promoter, and the promoter details. This results in a fold-change of the form
\begin{eqnarray}
FC &=& \left( 1 + \frac{k_R^+}{k_R^-} \rho \right)^{-1},\\
&=& (1 + \exp(-\Delta F_R + \log(\rho) ))^{-1},
\end{eqnarray}
where $\Delta F_R \equiv -\log(k_R^+/k_R^-)$, and the functional forms of $\rho$
for each model change as shown in Figure~\ref{fig1:means_cartoons}(C). Another
intuitive way to think about these two terms is as follows: in all kinetic
models shown in Figure~1(C) the repressor-bound state can only be reached from a
single repressor-free state. The ratio of these two states --repressor-bound and
adjacent repressor-free state-- must remain the same for all models, regardless
of the details included in other promoter states if $\Delta F_R$ represents an
effective free energy of the repressor binding the DNA operator. The presence of
other states then draws probability density from the promoter being in either of
these two states, making the ratio between the repressor-bound state and
\textit{all} repressor-free states different. The log difference in this ratio
is given by $\log(\rho)$. Since model 1 and model 5 of Figure~1(C) consist of a
single repressor-free state, $\rho$ is then necessarily 1 (See
Appendix~\ref{sec:non_bursty} for further details).

The key outcome of our analysis of the models in
Figure~\ref{fig1:means_cartoons} is the existence of a master curve shown in
Figure~\ref{fig1:means_cartoons}(D) to which the fold-change predictions of all
the models collapse. This master curve is parametrized by only two effective
parameters: $\Delta F_R$, which characterizes the number of repressors and their
binding strength to the DNA, and $\rho$, which characterizes all other features
of the promoter architecture. The key assumption underpinning this result is
that no transcription occurs when a repressor is bound to its operator. Given
this outcome, i.e., the degeneracy of the different models at the level of
fold-change, a mean-based metric such as the fold-change that can be readily
measured experimentally is insufficient to discern between these different
levels of coarse-graining. The natural extension that the field has followed for
the most part is to explore higher moments of the gene expression distribution
in order to establish if those contain the key insights into the mechanistic
nature of the gene transcription process~\cite{Iyer-Biswas2009,Munsky2012}.
Following a similar trend, in the next section we extend the analysis of the
models to higher moments of the mRNA distribution as we continue to examine the
discriminatory power of these different models.