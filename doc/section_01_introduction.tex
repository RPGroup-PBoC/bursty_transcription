% !TEX root = ./busty_transcription.tex
\section{Introduction}

Gene expression presides over much of the most important dynamism 
of living organisms.   The level of expression of batteries of
different genes is altered as a result of spatiotemporal cues that integrate
chemical, mechanical and other types of signals.  The original repressor-operator model
conceived by Jacob and Monod in the context of bacterial
metabolism has now been transformed into the much
broader subject of gene regulatory networks in living organisms
of all kinds~\cite{Jacob1961, Britten1969, Ben-TabouDe-Leon2007}.  One of the remaining outstanding challenges
to have emerged in the genomic era is our continued
inability to predict the regulatory consequences of
different regulatory architectures, stemming first and foremost
from our ignorance about what those architectures even are with
more than 60\% of the genes even in an ostensibly well understood organism
such as  {\it E. coli} with no regulatory
insights at all~\cite{Rydenfelt2014-2,Belliveau2018,Ghatak2019, Santos_Zavaleta2019}.
But even once we have established the identity of key transcription
factors and their binding sites of a given promoter architecture,  there remains the predictive challenge
of understanding its input-output properties, an objective that can be met
by a myriad of approaches using the tools of statistical physics~\cite{Ackers1982, Shea1985, Buchler2003,Vilar2003a,Vilar2003b, Bintu2005a,Bintu2005c, Gertz2009,Sherman2012, Saiz2013,
Ko1991,Peccoud1995,Record1996,
Kepler2001, Sanchez2008,Shahrezaei2008,Sanchez2011, Michel2010}.  One root
to such predictive understanding is to focus on the simplest
regulatory architecture and to push the theory-experiment
dialogue as far and as hard as it can be pushed~\cite{Phillips2019}.  If we
demonstrate that we can pass that test by successfully predicting
both the means and variance in gene expression, then that provides
a more solid foundation upon which to launch into 
more complex problems.

To that end, in this paper we examine a wide variety of distinct models
for the simple repression regulatory architecture.  All of these
models coarse-grain away some of the important microscopic
features of these architectures that have been elucidated
by generations of genetics, molecular biology and biochemistry.  
Our aim is to understand the underlying principles of such
coarse-graining such that future models of regulatory response
can serve the powerful predictive role needed to take synthetic
biology from what is often a brilliant exercise in enlightened 
empiricism and recast it as the same kind of rational design process
we are used to in other branches of engineering.  
More precisely, we want phenomenology in the sense of
coarse-graining away atomistic detail, but still retaining biophysical meaning.
For example, we are not satisfied with the strictly phenomenological
approach offered by 
 Hill functions since we are motivated by studies like~\cite{Razo-Mejia2020}, for which Hill functions are
clearly insufficient since each new situation
requires a completely new set of parameters. Such work requires a quantitative theory of how biophysical
changes at the molecular level propagate to input-output functions at the
genetic circuit level. We want concepts, not mere facts. In particular a key
question is: at this level of coarse graining, what microscopic details do we
need to explicitly model, and how do we figure that out? For example, do we need
to worry about all or even any of the steps that individual RNAPs go through
each time they make a transcript? Turning the question around, can we see any
imprint of those processes in the available data? If the answer is no, then
those processes are irrelevant for our purposes. Forward modeling and inverse
(statistical inferential) modeling can complement each other beautifully here.

Figure~\ref{fig1:means_cartoons}(A) shows the qualitative picture of simple repression that
is implicit in the repressor-operator model.
An operator, the binding site on the DNA for a repressor protein, may be found
occupied by a repressor, in which case transcription is blocked from occurring.
Alternatively, that binding site may be found unoccupied, in which case RNA polymerase (RNAP) may bind and
transcription can proceed.
The key assumption we make in this simplest incarnation of
the repressor-operator model is that binding of repressor and RNAP in the promoter
region of interest is exclusive, meaning that one or the other may bind, but
never may both be simultaneously bound. It is often imagined that when the
repressor is bound to its operator, RNAP is sterically blocked from binding to
its promoter sequence. Current evidence suggests this is sometimes, but not
always the case, and it remains
an interesting open question precisely how a repressor bound far upstream is
able to repress transcription. 
\marginpar{\it Cite that fig, from ??Nathan or Bill or Rob??, showing
locations of repressor binding sites far upstream of promoter}
Suggestions include ``action-at-a-distance''
mediated by kinks in the DNA, formed when the repressor is bound, that prevent
RNAP binding. Nevertheless, our modeling in this work is sufficiently
coarse-grained that we simply assume exclusive binding and leave explicit accounting
of these details out of the problem.

\begin{figure}%[h!]
\centering
\includegraphics[width=\textwidth]{../figures/fig1/fig1.ai}
\caption{\textbf{An overview of the simple repression motif at the level of
means.} (A) Schematic of the qualitative biological picture of simple repression. (B)
A variety of possible mathematicized cartoons of simple repression, along
with the effective parameter $\rho$ which subsumes all regulatory details of the
architecture that do not directly involve the repressor. (C) The ``master curve'' to which
all cartoons in (B) collapse.}
\label{fig1:means_cartoons}
\end{figure}

The logic of the remainder of the paper is as follows.
In section~\ref{section_02_means.tex}, we show how
both thermodynamic models and kinetic models based upon
the chemical master equation all culminate in the same underlying
functional form for the fold-change in the average level of gene
expression as shown in Figure~\ref{fig1:means_cartoons}(C).  Section~\ref{section_03_beyond_means.tex} goes
beyond an analysis of the means by asking how the same models
presented in Figure~\ref{fig1:means_cartoons} can be used
to explore noise in gene expression. Of course, to make contact
with experiment, all of these models must make a commitment
to some numerical values for the key parameters found in each
such model and in Section~\ref{section_04_bayesian_inference.tex}
we explore the use of Bayesian inference to establish
these parameters and to rigorously answer the question of how
to discriminate between the different models.


%
%\mmnote{Key ideas, no particular order, that I haven't written down before:
%\begin{itemize}
%\item Our goal is to build phenomenological models of input-output functions of
%genetic circuits. More precisely, we want phenomenology in the sense of
%coarse-graining away atomistic detail, but still retaining biophysical meaning,
%e.g., we don't want to coarse-grain as far as Hill functions. Why not? We are
%motivated by studies like~\cite{Razo-Mejia2020}, for which Hill functions are
%insufficient. Such work requires a quantitative theory of how biophysical
%changes at the molecular level propagate to input-output functions at the
%genetic circuit level. We want concepts, not mere facts. In particular a key
%question is: at this level of coarse graining, what microscopic details do I
%need to explicitly model, and how do we figure that out? For example, do I need
%to worry about all or even any of the steps that individual RNAPs go through
%each time they make a transcript? Turning the question around, can we see any
%imprint of those processes in the available data? If the answer is no, then
%those processes are irrelevant for our purposes. Forward modeling and inverse
%(statistical inferential) modeling can complement each other beautifully here.
%\end{itemize}}

%Biochemistry tells us of a progression of steps, but theory tells us
%that coarse graining can be exact



