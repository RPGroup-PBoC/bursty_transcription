% !TEX root = ./busty_transcription.tex
\section{Introduction}
\mmnote{Key ideas, no particular order, that I haven't written down before:
\begin{itemize}
\item Our goal is to build phenomenological models of input-output functions of
genetic circuits. More precisely, we want phenomenology in the sense of
coarse-graining away atomistic detail, but still retaining biophysical meaning,
e.g., we don't want to coarse-grain as far as Hill functions. Why not? We are
motivated by studies like~\cite{Razo-Mejia2020}, for which Hill functions are
insufficient. Such work requires a quantitative theory of how biophysical
changes at the molecular level propagate to input-output functions at the
genetic circuit level. We want concepts, not mere facts. In particular a key
question is: at this level of coarse graining, what microscopic details do I
need to explicitly model, and how do we figure that out? For example, do I need
to worry about all or even any of the steps that individual RNAPs go through
each time they make a transcript? Turning the question around, can we see any
imprint of those processes in the available data? If the answer is no, then
those processes are irrelevant for our purposes. Forward modeling and inverse
(statistical inferential) modeling can complement each other beautifully here.
\end{itemize}}

\fig{fig1:means_cartoons}A shows the qualitative picture of simple repression.
An operator, the binding site on the DNA for a repressor protein, may be found
occupied by a repressor, in which case transcription is blocked from occurring.
Or it may be found unoccupied, in which case RNA polymerase (RNAP) may bind and
transcription can proceed.

The key assumption we make is that binding of repressor and RNAP in the promoter
region of interest is exclusive, meaning that one or the other may bind, but
never may both be simultaneously bound. It is often imagined that when the
repressor is bound to its operator, RNAP is sterically blocked from binding to
its promoter sequence. Current evidence suggests this is sometimes, but not
always the case~\mmnote{Cite that fig, from ??Nathan or Bill or Rob??, showing
locations of repressor binding sites far upstream of promoter}, and it remains
an interesting open question precisely how a repressor bound far upstream is
able to repress transription. Suggestions include ``action-at-a-distance''
mediated by kinks in the DNA, formed when the repressor is bound, that prevent
RNAP binding. Nevertheless, our modeling in this work is sufficiently
coarse-grained that we simply assume exclusive binding and need not worry about
such precise details.

\mmnote{Add a notation aside somewhere that all association rates are written as zeroth-order rates. In other words, the concentration of the molecule is hidden inside, so for instance, $k_R^+$ is proportional to repressor copy number.}