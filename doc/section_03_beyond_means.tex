% !TEX root = ./busty_transcription.tex
\section{Beyond means}
\subsection{A one-state promoter with bursting is the ``best'' model of constitutive promoters}
Before we can tackle simple repression, we need an adequate phenomenological
model of constitutive expression. The literature abounds with options from which
we can choose, and we show several potential kinetic models for constitutive
promoters in~\fig{fig:constit_cartoons}. Let us consider the suitability of each
model for our purposes in turn.

\begin{figure}%[h!]
\centering
\includegraphics[width=0.9\textwidth]{../figures/fig1.5/fig1point5.ai}
\caption{\textbf{Constitutive promoter cartoon comparison.}
The left column depicts various plausible cartoons for constitutive promoters.
We also list literature references where models have been used previously. In
model (1), transcripts are produced in a Poisson process~\cite{Jones2014}. Model
(2) features explicit modeling of RNAP binding/unbinding
kinetics~\cite{Phillips2015a}. Model (3) is a more detailed generalization of
model (2), treating transcription initiation as a multi-step process proceeding
through closed and open complexes~\cite{Mitarai2015}. Model (4) is somewhat
analogous to (2) except with the precise nature of active and inactive states
left ambiguous~\cite{Razo-Mejia2020}. Finally, model (5) can be viewed as a
certain limit of model (4) in which transcripts are produced in bursts, and
initiation of bursts is a Poisson process. \mmnote{Haven't found model 5
formulated quite like this in the literature, usually burstiness is handled like
model 4, but I should probably search a bit more.} The right column shows the
Fano factor for each model. Note especially the crucial diagnostic: (2) and (3)
have $\nu$ strictly below 1, while only for (4) and (5) can $\nu$ exceed 1.
Models with Fano factors $\le 1$ cannot produce the FISH data observed
in~\cite{Jones2014} without introducing additional assumptions and model
complexity.}
\label{fig:constit_cartoons}
\end{figure}

\subsubsection{Model 1 - Poisson promoter.}
This is the picture from Jones et.\ al.~\cite{Jones2014}, which assumes that
transcripts are produced as a Poisson process from a single promoter state. This
model insists that the ``true'' mRNA distribution is Poisson, implying the Fano
factor $\nu$ must be 1. In~\cite{Jones2014}, the authors carefully attribute
deviations from Fano = 1 to intensity variability in fluorescent spots, gene
copy number variation, and copy number fluctuations of the transcription
machinery, e.g., RNAP itself. In this picture, the master equation makes no
appearance, and all the corrections to Poisson behavior are derived as additive
corrections to the Fano factor. For disproving the ``universal noise curve''
from So et al~\cite{So2011}, this picture was excellent. It is appealing in its
simplicity, with only two parameters, the initiation rate $r$ and degradation
rate $\gamma$. Since $\gamma$ is independently known from other experiments, and
the mean mRNA copy number is $r/\gamma$, $r$ is easily inferred from data. In
other words, the model is not excessively complex for the data at hand. But for
many interesting questions, for instance in the recent work~\mmnote{cite
Manuel's preprint once posted}, knowledge of means and variances alone is
insufficient, and a description of the full distribution of molecular counts is
necessary. For this we need a( slightly) more complex model than model 1 that
would allow us to incorporate the non-Poissonian features of constitutive
promoters directly into a master equation formulation.

\subsubsection{Model 2 - Two-state promoter, RNAP bound or unbound.}
This model was considered in, e.g.,~\cite{Phillips2015a} and
~\cite{Phillips2019}. Here transcription initiation proceeds from the bound to
the unbound state, reflecting the microscopic reality that an RNAP that has
begun to elongate a transcript is no longer available at the start site to begin
another. The problem with this picture is that the Fano factor is
\begin{align}
    \nu = 1 -
        \frac{r\rate{k}{P}{+}}
            {\left(\rate{k}{P}{+} + \rate{k}{P}{-} + r\right)
             \left(\gamma + \rate{k}{P}{+} + \rate{k}{P}{-} + r\right)},
\end{align}
which is always $<1$. To make contact with the experimental reality of $\nu>1$,
we will have to do, at a minimum, the same corrections to Poisson behavior as in
model 1 above. So while this model adds an appealing element of microscopic
reality, we are forced to reject it as the additional complexity is unable to
capture the phenomenology of interest. Obviously the promoter state does in fact
proceed through cycles of RNAP binding, initiating, and elongating, but it seems
that the super-Poissonian noise in mRNA copy number we want to model must be
governed by other features of the system.

\subsubsection{Model 3 - Three-state promoter, multistep transcription
initiation and escape.} How might we remedy the deficits of model 2? It is
known~\cite{DeHaseth1998} that once RNAP initially binds the promoter region, a
multitude of distinct steps occur sequentially before RNAP finally escapes into
the elongation phase. Perhaps adding some of this mechanistic detail might
rescue model 2? The next simplest refinement of model 2 could consider open
complex formation and promoter escape as separate steps rather than as a single
effective step. In other words, we construct model 3 by adding a single extra
state to model 2, and we will label the two RNAP-bound states as the closed and
open complexes, despite the true biochemical details certainly being more
complex. The authors of~\cite{Mitarai2015}, for instance, considered this model,
although they added an additional repressor bound state and did not explicitly
consider the limit with no repressor that we analyze here. Again, our goal here
is not a complete accounting of all the relevant biochemical detail; this is an
exploratory search for the important features our model needs to include.

Unfortunately, as the authors of~\cite{Mitarai2015} hint, this model too has
Fano factor $\nu<1$. This can be seen by starting with their Eq.~A1 for the Fano
factor of the analogous model with repressor and taking the limit as repressor
concentration goes to zero. After substantial algebra, the result is
\begin{align}
\nu = 1 - \frac{r k_O k_P^+}{\mathcal{Z}}
\frac{k_P^+ + k_P^- + k_O + \gamma}
        {\mathcal{Z} + \gamma(k_P^+ + k_P^- + k_O + \gamma) + \gamma^2},
\end{align}
where we defined $\mathcal{Z} = r(k_O + k_P^-) + k_P^+(k_O + r)$ for notational
tidiness. This is necessarily less than 1 for arbitrary rate constants.

In fact, we suspect \textit{any} model in which transcription proceeds through a
multistep cycle must necessarily have $\nu<1$. The intuitive argument compares
the waiting time distribution to traverse the cycle with the waiting time for a
Poisson promoter (model 1) with the same mean time. The latter is simply an
exponential distribution. The former is a convolution of multiple exponentials,
and intuitively should be more peaked with a smaller fractional width than an
exponential with the same mean.\footnote{This can be made more precise with a
result from~\cite{Stewart2007}, who showed that the convolution of multiple
gamma distributions (of which the exponential distribution is a special case) is
to a very good approximation also gamma distributed. Using their Eq.~2 for the
distribution of the convolution, with shape parameters set to 1 to give
exponential distributions, the total waiting time distribution has a ratio of
variance to squared mean $\sigma^2/\mu^2 = \sum_i k_i^2/\left(\sum_i
k_i\right)^2$, where the $k_i$ are the rates of the individual steps. Clearly
this is less than 1 and therefore the total waiting time distribution is
narrower than the corresponding exponential.} A less disperse waiting time
distribution means transcription initations are more uniformly distributed in
time relative to a Poisson process. Hence the distribution of mRNA over a
population of cells should be less variable compared to Poisson, giving
$\nu<1$.\footnote{This last step, while intuitive, can be argued by analogy to
photon statistics where antibunching gives rise to sub-Poissonian
noise~\cite{Paul1982, Zou1990}. Although loopholes exist, we do not expect they
apply for our problem. Nevertheless we refrain from elevating this
cycle/sub-Poissonian equivalence to a ``theorem.'' \mmnote{Antibunching is the
``obvious'' analogy for me but it'd be totally out of left field for most
readers, any suggestions on a more comprehensible reference??}}

Regardless of the merits of this model in describing the noise properties of
constitutive transcription initation, it does not capture the dominant source of
noise we want to include in our phenomenological model, so we must discard these
details and search elsewhere.

\subsubsection{Model 4 - Two-state promoter, ``active'' and ``inactive''
states.} Inspired by~\cite{Razo-Mejia2020}, we revisit an analog of model 2, but
the interpretation of the two states is changed. Rather than explicitly viewing
them as RNAP bound and unbound, we view them as ``active'' and ``inactive,''
which are able and unable to initiate transcripts, respectively. We are
noncommittal as to the microscopic details of these states.

One interpretation~\cite{Chong2014, Sevier2016} for the active and inactive
states is that it represents the promoter's supercoiling state: transitions to
the inactive state are caused by accumulation of positive supercoiling, which
inhibits transcription, and transitions back to ``active'' are caused by gyrase
or other topoisomerases relieving the supercoiling. This is an interesting
possibility because it would mean the timescale for promoter state transitions
is driven by topoisomerase kinetics, not by RNAP kinetics. From in vitro
measurments, the former are suggested to be of order minutes~\cite{Chong2014}.
Contrast this with model 2, where the state transitions are assumed to be
governed by RNAP, which, assuming a copy number per cell of order $10^3$, has a
diffusion-limited association rate $k_{on} \sim 10^2~\text{s}^{-1}$ to a target
promoter. Combined with known $K_D$'s of order $\mu$M, this gives an RNAP
dissociation rate $k_{off}$ of order $10^2$. As we will show below, however,
there are some lingering puzzles with interpreting this supercoiling
interpretation, so we leave it as a speculation and refrain from assigning
definite physical meaning to the two states in this model.

Intuitively one might expect that, since transcripts are produced as a Poisson
process only when the promoter is in one of the two states in this model,
transcription initiations should now be ``bunched,'' in constrast to the
``anti-bunching'' of models 2 and 3 above. One might further guess that this
bunching would lead to super-Poissonian noise in the mRNA distribution over a
population of cells. An honest calculation of the Fano factor produces
\begin{align}
\nu &= 1 + \frac{r k^-}{(k^+ + k^- + \gamma)(k^+ + k^-)},
\end{align}
which is strictly greater than 1, verifying the above intuition. Note we have
dropped the $P$ label on the promoter switching rates to emphasize that these
very likely do not represent kinetics of RNAP itself. This calculation can also
be sidestepped by noting that the model is mathematically equivalent to the
simple repression repression model from~\cite{Jones2014}, with states and rates
relabeled and reinterpreted.

How does this model compare to model 1 above? In model 1, all non-Poisson
features of the mRNA distribution were handled as extrinsic corrections. By
contrast, here the 3 parameter model is used to fit the full mRNA distribution
as measured in mRNA FISH experiments. In essense, all variability in the mRNA
distribution is regarded as ``intrinsic,'' arising either from stochastic
initiation or from switching between the two coarse-grained promoter states. The
advantage of this approach is that it fits neatly into the master equation
picture, and the parameters thus inferred can be used as input for more
complicated models with regulation by transcription factors.

While this seems promising, there is a major drawback for our purposes which was
already uncovered by the authors of~\cite{Razo-Mejia2020}: the statistical
inference problem is nonidentifiable, in the sense described in Section 4.3
of~\cite{Gelman2013}, meaning it is impossible to infer the parameters $r$ and
$k^-$ from the FISH data of~\cite{Jones2014} (as shown in Fig.~S2
of~\cite{Razo-Mejia2020}). Rather, only the ratio $r/k^-$ could be inferred. In
that work, the problem was worked around with an informative prior on the ratio
$k^-/k^+$. That approach is unlikely to work here, as, recall, our entire goal
in modeling constitutive expression is to use it as the basis for a yet more
complicated model, when we add on repression. But adding more complexity to a
model that is already poorly identified is a fool's errand, so we will explore
one more potential model.

\subsubsection{Model 5 - One-state promoter with explicit bursts.}
This model is inspired by the failure mode of model 4. The key observation above
was that, as found in~\cite{Razo-Mejia2020}, only two parameters, $k^+$ and the
ratio $r/k^-$, could be directly inferred from the FISH data
from~\cite{Jones2014}. So let us take this seriously and imagine a model where
these are the only two model parameters. What would this model look like?

To develop some intuition, consider model 4 in the limit $k^+ \ll k^- \lesssim
r$, which is roughly satisfied by the parameters inferred
in~\cite{Razo-Mejia2020}. In this limit, the system spends the majority of its
time in the inactive state, occasionally becoming active and making a burst of
transcripts. This should call to mind the well-known phenomenon of
transcriptional bursting, as reported in,
e.g.,~\cite{Golding2005,Chong2014,Sevier2016}\mmnote{should probably add some
more cites here}. Let us make this correspondence more precise. The mean dwell
time in the active state is $1/k^-$. While in this state, transcripts are
produced at a rate $r$ per unit time. So on average, $r/k^-$ transcripts are
produced before the system switches to the inactive state. Once in the inactive
state, the system dwells there for an average time $1/k^+$ before returning to
the active state and repeating the process. $r/k^-$ resembles an average burst
size, and $1/k^+$ resembles the time interval between burst events. More
precisely, $1/k^+$ is the mean time between the end of one burst and the start
of the next, whereas $1/k^+ + 1/k^-$ would be the mean interval between the
start of two successive burst events, but in the limit $k^+ \ll k^-$, $1/k^+ +
1/k^- \approx 1/k^+$. Note that this limit ensures that the waiting time between
bursts is approximately exponentially distributed, with mean set by the only
timescale left in the problem, $1/k^+$.\footnote{If instead it were the case
that $k^+ \sim k^-$, then the waiting time $1/k^+ + 1/k^-$ would have a peaked
distribution, but this does not appear to be the case for any of datasets
from~\cite{Jones2014}.}

Let us now verify this intuition with a precise derivation to check that $r/k^-$
is in fact the mean burst size and to obtain the full burst size distribution.
Consider first a constant, known dwell time $T$ in the active state. Transcripts
are produced at a rate $r$ per unit time, so the number of transcripts $n$
produced during $T$ fits exactly the ``story'' for the Poisson distribution,
i.e.,
\begin{equation}
    P(n\mid T) = \frac{(rT)^n}{n!} \exp(-rT).
\end{equation}
Since the dwell time $T$ is unobservable, we actually want $P(n)$, the dwell
time distribution with no conditioning on $T$. Basic rules of probability theory
tell us we can write $P(n)$ in terms of $P(n\mid T)$ as
\begin{equation}
    P(n) =\int_0^\infty P(n\mid T) P(T) dT.
\end{equation}
But we know the dwell time distribution $P(T)$, which is exponentially
distributed according to
\begin{equation}
    P(T) = k^- \exp(-T k^-),
\end{equation}
so $P(n)$ can be written as
\begin{equation}
    P(n) = k^- \frac{r^n}{n!}
            \int_0^\infty T^n\exp[-(r + k^-)T]\,dT.
\end{equation}
A standard integral table shows $\int_0^\infty x^n e^{-ax}\,dx = n!/a^{n+1}$, so
\begin{equation}
    P(n) = \frac{k^- r^n}{(k^- + r)^{n+1}}
        = \frac{k^-}{k^- + r}
            \left(\frac{r}{k^- + r}\right)^n
        = \frac{k^-}{k^- + r}
            \left(1 - \frac{k^-}{k^- + r}\right)^n,
\end{equation}
which is exactly the geometric distribution with standard parameter
$\theta\equiv k^-/(k^- + r)$ and domain $n \in \{0, 1, 2, \dots\}$ (this is one
of two common conventions for the geometric distribution). The mean of the
geometric distribution, with this convention, is
\begin{align}
\langle n\rangle = \frac{1 - \theta}{\theta}
        = \left(1 - \frac{k^-}{k^- + r}\right)
                    \frac{k^- + r}{k^-}
        = \frac{r}{k^-},
\end{align}
exactly as we guessed intuitively above.

So in taking the limit $r,k^-\rightarrow\infty$, $r/k^-\equiv b$, we obtain a
model which effectively has only a single promoter state, which initiates bursts
at rate $k^+$ (transitions to the active state, in the model 4 picture). The
master equation for mRNA copy number $m$ takes the form
\begin{align}
\begin{split}
\deriv{t}{p(m,t)} = & (m+1)\gamma p(m+1,t) - m\gamma p(m,t) \\
        &+ \sum_{j=0}^{m-1} k^+ p(j,t) Geom(m-j;b)
         - \sum_{j=m+1}^\infty k^+ p(m,t) Geom(j-m;b),
\end{split}
\end{align}
where $Geom(n;b)$ is the geometric distribution with mean~$b$, i.e., $Geom(n;b)
= \frac{1}{1+b}\left(\frac{b}{1+b}\right)^n$ (with domain over nonnegative
integers as above). The first two terms are the usual mRNA degradation terms.
The third term enumerates all ways the system can produce a burst of transcripts
and arrive at copy number $m$, given that it had copy number $j$ before the
burst. The fourth term allows the system to start with copy number $m$, then
produce a burst and end with copy number $j$. In fact this last sum has trivial
$j$ dependence and simply enforces normalization of the geometric distribution.
Carrying it out we have
\begin{equation}
\begin{split}
\deriv{t}{p(m,t)} = & (m+1)\gamma p(m+1,t) - m\gamma p(m,t) \\
        &+ \sum_{j=0}^{m-1} k^+ p(j,t) Geom(m-j;b)
            - k^+ p(m,t),
\end{split}
\end{equation}
This improves on model 4 in that now the parameters are easily inferred, as we
will see later, and have clean interpretations. The non-Poissonian features are
attributed to the emprically well-established phenomenological picture of bursty
transcription.

The big approximation in going from model 4 to 5 is that a burst is produced
instantaneously rather than over a finite time. If the true burst duration is
not short compared to transcription factor kinetic timescales, this could be a
problem in that mean burst size in the presense and absence of repressors could
change, rendering parameter inferences from the constitutive case inappropriate.
Let us make some simple estimates of this.

Consider the time delay between the first and final RNAPs in a burst initiating
transcription (\textit{not} the time to complete transcripts, which potentially
could be much longer.) If this timescale is short compared to the typical search
timescale of transcription factors, then all is well. The estimates from
deHaseth et.\ al.~\cite{DeHaseth1998} put RNAP's diffusion-limited on rate
around $\sim\text{few}\times10^{-2}~\text{nM}^{-1}~\text{s}^{-1}$ and polymerase
loading as high as $1~\text{s}^{-1}$. Then for reasonable burst sizes of $<10$,
it is reasonable to guess that bursts might finish initiating on a timescale of
tens of seconds or less (with another 30-60 sec to finish elongation, but that
does not matter here). A transcription factor with typical copy number of order
10 (or less) would have a diffusion-limited association rate of order
$(10~\text{sec})^{-1}$. Higher copy number TFs tend to have many binding sites
over the genome, which should serve to pull them out of circulation and keep
their effective association rates from rising too large. Therefore, there is
\textit{perhaps} a timescale separation possible between transcription factor
association rates and burst durations, but this assumption could very well break
down, so we will have to keep it in mind when we infer repressor rates from the
Jones et.\ al.\ FISH data later.

In reflecting on these 5 models, the reader may feel that exploring a multitude
of potential models just to return to a very minimal phenomenological model of
bursty transcription may seem highly pedantic. But the purpose of the exercise
was to examine a host of models from the literature and understand why they are
insufficient, one way or another, for our purposes. Along the way we have
learned that the detailed kinetics of RNAP binding and initiating transcription
are probably irrelevant for setting the population distribution of mRNA. The
timescales are simply too fast, and as we will see later
in~\fig{fig:constit_post}, the noise seems to be governed by slower timescales.
Perhaps in hindsight this is not surprising: intuitively, the degradation rate
$\gamma$ sets the fundamental timescale for mRNA dynamics, and any other
processes that substantially modulate the mRNA distribution should not differ
from $\gamma$ by orders of magnitude.